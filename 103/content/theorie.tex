\section{Zielsetzung}
\label{Zielsetzung}
In diesem Versuch soll das Elastizitätsmodul von verschiedenen Stoffen ermittelt werden.
Diese unterscheiden sich in Gestalt, Metall und Legierung.

\section{Theorie}
\label{Theorie}
Wenn Spannungen bzw. Druck auf eine Oberfläche wirken, kommt es zu Oberflächen- und/oder Volumenänderungen.
Die Normalspannung $\sigma$ ist die senkrechte Komponente dieser Spannung zur Oberfläche.
Wenn die Längenänderung $\Delte L$ zur Körperdimension $L$ relativ klein ist, kann der Zusammenhang zwischen $\frac{\Delta L}{L}$ und der Spannung $\sigma$ als linear angesehen werden, auch als Hookschens Gesetz bekannt:

\begin{equation}
    \sigma = E  \cdot \frac{\Delta L}{L} .
\end{equation}

Das Elastizitätsmodul E ist die Materialkonstante eines Werkstoffes.
Diese kann durch abmessen der Längenänderung $\Delta L$ ermittelt werden.
In diesem Versuch wird die Berechnung von $E$ durch die Biegung von zwei Metallstäben realisiert.
Hierbei nutzt man das Drehmoment $M_F$ aus, das durch die angreifende Kraft $F$ an einer Stelle $x$ des Stabes verursacht wird.
Das Drehmoment verschiebt den Querschnitt $Q$ aus seiner Ausgangslage.
Dabei wird bei einer einseitigen Einspannung die obere Schicht ausgedehnt und die untere gestaucht.
Es enstehen Zug- und Druckspannungen, die entgegenwirken.
Dazwischen befindet sich die sogennante neutrale Faser.
Diese Fläche behählt bei der Biegung ihre ursprüngliche Länge bei.
Durch die entgegengesetzten Spannungen kommt es zu eienm Gleichgewichtszustand und schließlich zu einer endlichen Dehnung $D$.

Dieser Gleichgewichtszustand ist über das äußere Drehmoment $M_F$ und das innere $M_\sigma$ definiert:

\begin{equation}
    M_F = F \cdot (L - x) ,
\end{equation}
\begin{equation}
    M_\sigma = \int_{Q}^{} y \cdot \sigma(y)\, dq .
\end{equation}

$y$ ist der Abstand des Flächenelementes $dq$ von der neutralen Faser $x$.

Bei einer einseitigen Einspannung ergibt sich für die Dehnung $D$ in Abhängigkeit vom Abstand $x$ von der Einspannung:
\begin{equation}
    D(x) = \frac{F}{2 \cdot E \cdot I} \cdot (L \cdot x^2 - \frac{x^3}{3}) .
\end{equation}
$I$ steht für das Flächenträgheitsmoment des Querschnitts $Q$.

Bei einer zweiseitigen Auflage des Stabes gelten folgende Gleichungen.

Für $0 \leq x \leq \frac{L}{2}$ gilt 
\begin{equation}
    D(x) = \frac{F}{48 \cdot E \cdot I} \cdot (3 \cdot L^2 \cdot x - 4 \cdot x^3)
\end{equation}
und für $\frac{L}{2} \leq x \leq L$
\begin{equation}
    D(x) = \frac{F}{48 \cdot E \cdot I} \cdot (4 \cdot x^3 - 12 \cdot L \cdot x^2 + 9 \cdot L^2 \cdot x - L^3) .
\end{equation}
