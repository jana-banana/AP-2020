\section{Diskussion}
\label{sec:Diskussion}

Die Versuchsdurchführung verlief ohne größere Probleme.
Da die Messungen stark von dem Blickwinkel auf den Leuchtschirm beeinflusst werden, können hierbei einige Messunsicherheiten nicht vermieden werden.
Auch die Justierung der Y-Ablenkung erfolgte nur mündlich, da eine der Praktikantinnen den Leuchtschirm im Blick hatte und die andere den Regler verstellte.
\\
\noindent
Für das Verhältnis $\frac{p \cdot L}{2\cdot d}$ besteht eine 43,9\% Abweichung zwischen Theorie- und ermittelten Wert.
Die oben genannten Bedingungen haben sicherlich einen Einfluss auf das Ergebnis, dennoch ist von einem systematischen Fehler auszugehen, da diese Abweichung dafür viel zu groß ist.
Eine Vermutung aufgrund des Alters der Messaparatur ist, dass die Kathodenstrahlröhre nicht mehr vollständig evakuiert ist, sodass doch größere Wechselwirkungen mit Luftmolekülen zu Stande kommen.
Während des Versuches kam es auch mehrmals dazu, dass der Leuchtbildschirm keine Auftreffpunkte angezeigt hat, was auch auf einen systematischen Fehler deuten kann.
\\
\noindent
Die Bestimmung der Sinusfrequenz verläuft nach Verwerfen des dritten Messpaares eindeutig, somit beträgt $\nu_\text{Sin} = \SI{50}{\hertz}$.
\noindent
Der zweite Teil, der Versuch zur Untersuchung von Elektronen im B-Feld, verlief ähnlich.
Einige Messunsicherheiten können, wie oben beschrieben, nicht vermieden werden.
Die Diagramme zeigen dennoch ein gutes Ergebnis, da die erwartete Proportionalität zu sehen ist.
Eine Ablenkung bei der Untersuchung des Erdmagnetfeldes wurde auch während des Versuches gut beobachtet.
Die experimentell bestimmte spezifische Ladung ist in derselben Größenordnung wie der Theoriewert, um genau zu sein im $\SI{e11}{\coulomb\per\kilo\gram}$-Bereich.
Mit $\frac{\symup{e_0}}{\symup{m_0}} = (3.6248\pm 0.0014) \cdot 10^{11} \, \si{\coulomb\per\kilo\gram}$ weicht dieser Wert um ca. 50\% vom Theoriewert ab.
Der ermittelte Wert für das Erdmagnetfeld $B = 3.5074 \cdot 10^{-5} \, \si{\tesla}$ ist auch knapp im $\si{\micro\tesla}$-Bereich.
Einige Gründe für diese Ungenauigkeiten sind zum einen die bereits erwähnten Bedingungen des Versuchsdurchlaufs, zum anderen der vermutete systematische Fehler.
Bei der Untersuchung der Elektronen im B-Feld wurde auch kein Deklinatorium-Inklinatorium verwendet.
DAdurch wurden die Himmelsrichtungen abgeschätzt, sodass über die verschiedenen Komponenten des Erdmagnetfeldes wenig ausgesagt werden kann.