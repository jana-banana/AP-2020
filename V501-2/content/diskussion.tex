\section{Diskussion}
\label{sec:Diskussion}

Die Versuchsdurchführung verlief ohne größere Probleme.
Da die Messung stark von dem Blickwinkel auf den Leuchtschirm beeinflusst werden, können hierbei einige Messunsicherheiten nicht vermieden werden.
Auch die Justierung der Y-Ablenkung erfolgte nur mündlich, da einer der Praktikantinnen den Leuchtschirm im Blick hatte und die andere den Regler verstellte.

\noindent
Für das Verhältnis $\frac{p \cdot L}{2\cdot d}$ besteht eine Abweichung zwischen Theorie- und ermittelten Wert von 43,9\%.
Die oben genannten Bedingungen haben sicherlich einen Einfluss auf das Ergebnis, dennoch ist von einem systematischen Fehler auszugehen, da diese Abweichung dafür viel zu groß ist.
Eine Vermutung aufgrund des Alters der Messaparatur ist, dass die Kathodenstrahlröhre nicht mehr vollständig evakuiert ist, sodass doch größere Wechselwirkungen mit Luftmolekülen zu Stande kommen.
Während des Versuches kam es auch mehrmals dazu, dass der Leuchtbildschirm keine Auftreffpunkte angezeigt hat, was auch auf einen systematischen Fehler deuten kann.

\noindent
Der zweite Teil, der Versuch zur Untersuchung von Elektronen im B-Feld, verlief ähnlich.
Einige Messunsicherheiten können, wie oben beschrieben, nicht vermieden werden.
Die Diagramme zeigen dennoch ein gutes Ergebnis.
Die erwartete Proportionalität ist sehen.
Eine Ablenkung bei der Untersuchung des Erdmagnetfeldes wurde auch während des Versuches gut gesehen.
Die ermittelten Werte sind auch im $\si{\micro\tesla}$-Bereich.