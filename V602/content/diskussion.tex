\section{Diskussion}
\label{sec:Diskussion}

Der experimentell gemessenen Winkel beträgt $\alpha_{\text{GM}} = \SI{28.2}{\degree}$, der Sollwinkel ist $\SI{28}{\degree}$. 
Also weicht der experimentelle Wert um $\SI{0.7}{\percent}$ ab. Dies ist ein recht gutes Ergebnis, welches auch gefordert werden muss. Denn würde die gemessenen
Position des Maximums mehr als $\SI{3}{\percent}$ vom Sollwinkel abweichen, würden die nachfolgenden Spektren nicht der $K$-Kante entsprechen. 

\noindent
In der \autoref{tab:klinie} sind die im Experiment ermittelten Werte der Energien der $K_{\alpha}$ und der $K_{\beta}$ Linie neben den entsprechenden Literaturwerten
und deren prozentualen Abweichung zu finden. Es fällt auf, dass die Werte eine sehr geringe Abweichung haben, obwohl keine nähere Messsung der charakteristischen
Linien durchgeführt wurden.
\begin{table}
    \centering
    \caption{Die Energien der $K_{\alpha}$ und der $K_{\beta}$ Linie im Vergleich.}
    \label{tab:klinie}
    \begin{tabular}{c c c c}
        \toprule
        Linie & $E_{K, \text{exp}} \, [\si{\electronvolt}] $ & $E_{K, \text{Lit}} \, [\si{\electronvolt}]$ \cite{E_abs}  & Abweichung [$\si{\percent}$] \\
        \midrule
        $\alpha$ & 8043 $\pm$ 34 & 8109 & 0.80 $\pm$ 0.40 \\ 
        $\beta$  & 8910 $\pm$ 40 & 8914 & 0.00 $\pm$ 0.05 \\  
        \bottomrule      
    \end{tabular}
\end{table}


\noindent
In der \autoref{tab:sigmavergleich} sind die im Experiment ermittelten Werte für die Abschirmkonstanten der verschiedenen Absorbermaterialien eingetragen, sowie die 
Literaturwerte und die prozentuale Abweichung. Generell sind die Abweichungen recht gering, die wenigen Abweichung kommen daher, dass die Winkel nur auf $\SI{0.1}{\degree}$ 
genau ermittelt werden können. Beispielsweise bei Rubidium als Absorptionsmaterial wurde der Winkel zufällig genau gemessen, sodass der Literaturwert sehr gut 
angenähert wird.
\begin{table}
    \centering
    \caption{Die im Experiment ermittelten und aus der Literatur bekannten Werte der Abschirmkonstante $\sigma$ im Vergleich.}
    \label{tab:sigmavergleich}
    \begin{tabular}{c c c c }
        \toprule
        Material & $\sigma_{\text{exp}} $ & $ \sigma_{\text{Lit}}$ \cite{sigma} & Abweichung [$\si{\percent}$]\\
        \midrule
        \text{Zn} & 3.60 $\pm$ 0.07 & 3.56 & 1.11 \\
        \text{Ga} & 3.67 $\pm$ 0.08 & 3.62 & 3.12 \\
        \text{Br} & 3.84 $\pm$ 0.12 & 3.84 & 0.00 \\
        \text{Rb} & 4.04 $\pm$ 0.14 & 3.95 & 2.19 \\
        \text{Sr} & 4.04 $\pm$ 0.16 & 3.99 & 1.31 \\
        \text{Zr} & 4.28 $\pm$ 0.18 & 4.39 & 2.52 \\
        \bottomrule        
    \end{tabular}
\end{table}

\noindent
Abschließend wurde mittels Linearer Regression die Rydbergenergie $R_{\infty} $ ermittelt.
Die Werte lauten:
\begin{align*}
    R_{\infty, \text{exp}} &= \SI{12.62(14)}{\electronvolt}\\
    R_{\infty, \text{Lit}} &= \SI{13.6}{\electronvolt}
\end{align*}
Die Abweichung ist $\SI{7.8}{\percent}$, welches daher kommt, dass dieser Wert angenähert ist durch bereits gerundete Werte und sich somit die Fehler fortgepflanzt haben.
Insgesamt ist zu sagen, dass die Abweichung der ermittelten Werte in Anbetracht der zur Auswertung benötigten Schätzungen, recht gering sind.
Da die eigentliche Messung mit Hilfe eines Röntgengerätes durchgeführt wurden, ist es unwahrscheinlich, dass es Fehlerquellen für systematische Fehler gibt.