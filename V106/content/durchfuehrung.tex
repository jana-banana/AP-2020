\section{Durchführung}
\label{sec:Durchführung}

Einaml lange 1 und dann lange 2

\subsection{Gleichsinnige Schwingung}
\label{subsec:gleich}
Um eine gleichsinnige Schwingung bestmöglich zu untersuchen, wird auf die Kopplungsfeder verzichtet.
Diese hat wie in der Therie beschrieben......... \autoref{} keinen Einfluss auf die Bewegung.
Dennoch ist eine gleichsinnige Schwingung mit einer Kopplungsfeder schwer umzusetzen.
\noindent
Zunächst wird das erste Pendel ausgelenkt, wobei auf die Kleinwinkelnäherung gachtet wird.
Die Schwingungsdauer für fünf Schwingungen wird mithilfe von Stoppuhren aufgenommen.
Durch die Aufnahme von fünf Schwingungen sollen Ungenauigkeiten minimiert werden.
Da beide Praktikanten gleichzeitig messen, werden 10 Messungen durchgeführt.
Somit wurden 20 Schwingungszeiten für ein Pendel notiert.
Derselbe Vorgang wird für das zweite Pendel durchgeführt.

\subsection{Gegensinnige Schwingung}
Für die gegensinnige Schwingung wird die Kopplungsfeder wieder eingesetzt.
Da die Auslenkungen nun gegesätzlich erfolgen, wird statt eine Auslenkung nach Außen (wie in \autoref{})
eine nach Innen, zur Mitte vorgenommen. Dadurch können Kollisionen der Masssen miteinander verhindert werden,
welche die Messungen unbrauchbar machen würden.
Ähnlich wie in \autoref{subsec:gleich} nehmen beide Praktikanten gleichzeitig Messungen auf, sodass 20 Schwingungszeiten notiert werden.
Hierbei werden für beide Pendel gleichzeitig die Schwingungsdauer aufgenommen.

\subsection{Gekoppelte Schwingung}
Bei der gekoppelten Schwingung wird zunächst eine Messung für die Schwingungsdauer vorgenommen.
Dabei wird das erste Pendel zur Ruhe gebracht und das zweite zum Schwingen angeregt.
Die Schwingungsdauer für zwei Schwingungen wird notiert.
Dieser Vorgang wird fünf Mal durchgeführt, sodass 10 Zeiten aufgenommen werden.
\noindent
Um die Schwebungsdauer zu ermitteln, wird wie zuvor beschrieben das erste Pendel zunächst in die Ruhelage versetzt und das zweite zum Schwingen
angeregt.
Das erste Pendel fängt an zu schwingen bis es sein Maximum erreicht und kehrt dann langsam wieder zurück zu seiner Ruhelage.
Diese Zeit zwischen den Stillständen wird von beiden Praktikanten gemessen.
Dieser Vorgang wird insgesamt fünf Mal wiederholt, sodass 10 Zeiten notiert werden.