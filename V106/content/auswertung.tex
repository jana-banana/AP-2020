\section{Auswertung}
\label{sec:Auswertung}

Es werden die gleichen Auswertungschritte für die verschiedenen Längen des Pendels durchgeführt. 
 
\subsection{Kurze Pendellänge}

  Das Pendel hat die Länge von $l = \SI{35}{\centi\metre}$. In der \autoref{tab:35cm} lassen sich die Messdaten finden. Es werden bei der Untersuchung der einzelnen
  Schwingungen sowie bei der gegensinnigen Schwingung $\num{5}$ Schwingungen gemessen, bei der gekoppelten Schwingung $\num{2}$ Schwingungsdauer und eine 
  Schwebungsdauer. Die Schwingungsdauern werden von beiden Praktikanten gleichzeitig gemessen, welches auch in der \autoref{tab:35cm} zu sehen ist. 

  \begin{table}
    \centering
    \caption{Die Messdaten von der Messung der Pendellänge $l = \SI{35}{\centi\metre}$.}
    \label{tab:35cm}
    \begin{tabular}{c c c c c}
      \toprule
      $ 5 \cdot T_{\text{rechts}} \mathbin{/} \si{\second} $ & $ 5 \cdot T_{\text{links}} \mathbin{/} \si{\second} $ & $ 5 \cdot T_{\text{gegen}} \mathbin{/} \si{\second} $ &
      $ 2 \cdot T_{\text{schwing}} \mathbin{/} \si{\second} $ & $ T_{\text{schweb}} \mathbin{/} \si{\second} $\\
      \midrule
      \multicolumn{5}{l}{Messdaten von Sonia}\\
      \midrule
      6.09 &  6.87 &   5.40 &   2.73 &   9.23   \\
      6.02 &  6.13 &   5.50 &   2.20 &   8.40   \\
      6.32 &  6.35 &   5.57 &   2.65 &   8.46   \\
      6.59 &  6.29 &   5.47 &   2.42 &   8.00   \\
      6.36 &  6.47 &   5.86 &   2.35 &   8.89   \\
      6.20 &  6.39 &   5.53 &        &          \\
      6.33 &  6.37 &   5.50 &        &          \\
      6.50 &  6.40 &   5.39 &        &          \\
      6.49 &  6.40 &   5.59 &        &          \\
      6.19 &  6.33 &   5.46 &        &          \\
      \midrule
      \multicolumn{5}{l}{Messdaten von Jana}\\
      \midrule
      6.58 &  6.45 &   5.36 &   2.40 &   8.86   \\
      6.36 &  6.16 &   5.16 &   2.53 &   8.02   \\
      6.22 &  6.13 &   5.32 &   2.26 &   8.23   \\
      6.10 &  6.18 &   5.38 &   2.23 &   7.93   \\
      6.46 &  6.01 &   5.56 &   2.30 &   9.00   \\
      6.23 &  6.13 &   5.30 &        &          \\
      6.20 &  6.13 &   5.35 &        &          \\
      6.22 &  6.18 &   5.32 &        &          \\
      6.22 &  6.32 &   5.32 &        &          \\
      6.22 &  6.49 &   5.36 &        &          \\
      \bottomrule
    \end{tabular}
  \end{table}

  \noindent Im folgen werden aus den aufgenommenen Daten die Mittelwerte $\bar{T} \pm \increment T$ für die jeweilige Schwingungsmessung gebildet. 
  Dies erfolgt über die Formeln
  \begin{align}  
    \bar{T} &= \frac{1}{N} \sum_{k=1}^{N} T_{\text{k}} \label{eqn:mittel}\\
    \increment T &= \frac{\sigma}{\sqrt{N}} , \label{eqn:fehlermittel} 
  \end{align}
  wobei $\sigma$ die Standardabweichung beschreibt. \\ \\

  \noindent Für die Schwingungsdauer der gleichsinnigen Schwingung wird erstmal die Schwingungsdauer der einzelnen Pendel als Mittelwert der gemessenen Zeiten
  genommen, aus diesen zwei Werten wird dann nochmal das Mittel gebildet. Für das rechte Pendel ergibt sich eine Schwingungsdauer von 
  \begin{equation*}
    \bar{T}_{\text{rechts}} = \SI{1.259(7)}{\second}, 
  \end{equation*}
  für das linke Pendel eine von 
  \begin{equation*}
    \bar{T}_{\text{links}} = \SI{1.262(8)}{\second}. 
  \end{equation*}
  Damit ist die experimentell ermittelte Schwingungsdauer der gleichsinnigen Schwingung $ T_{+} = \SI{1.260(5)}{\second}$. 
  
  \noindent Der Theoriewert der gleichsinnigen Schwingung berechnet sich durch die Formel \eqref{eqn:Tplus}. Der Wert ergibt sich zu 
  \begin{equation*}
    T_{\text{theo, +}} = \SI{1.187}{\second}.
  \end{equation*} \\ \\

  \noindent Die Dauer der gegensinnigen Schwingung wird über den Mittelwert $\bar{T}_{\text{gegen}}$ zu 
  \begin{equation*}
    T_{-} = \bar{T}_{\text{gegen}} = \SI{1.087(6)}{\second}
  \end{equation*} 
  bestimmt.

  \noindent Mit den Werten $T_{+}$ und $T_{-}$ lässt sich nun die Kopplungskonstante $\kappa$ über die Gleichung \eqref{eqn:kappa} berechnen. Der Fehler ergibt sich über die 
  Gauss'sche Fehlerfortpflanzung, welche für eine beliebige Funktion $f(x,y) $ 
  \begin{equation} \label{eqn:gauss}
    \increment f = \sqrt{(\increment x)^2 \cdot \left(\frac{\partial f}{\partial x}\right)^2 + (\increment y)^2 \cdot \left(\frac{\partial f}{\partial y}\right)^2}
  \end{equation}
  lautet. Für $\kappa $ folgt dementsprechend 
  \begin{equation} \label{eqn:fehlerk}
    \increment \kappa = \sqrt{(\increment T_{+})^2 \cdot \left( \frac{4 T_{+} T_{-}^2}{(T_{+}^2 + T_{-}^2)^2}\right)^2
                           (\increment T_{-})^2 \cdot \left( \frac{-4 T_{-} T_{+}^2}{(T_{+}^2 + T_{-}^2)^2}\right)^2 }. 
  \end{equation} 
  Somit ist die Kopplungskonstante 
  \begin{equation*}
    \kappa = \num{0.147(7)}.
  \end{equation*}

  \noindent Nun lässt sich der Theoriewert der gegensinnigen Schwingung nach der Formel \eqref{} berechnen. Er ergibt sich zu 
  \begin{equation*}
    T_{\text{theo, -}} = \SI{1.1696(8)}{\second}, 
  \end{equation*} 
  wobei sich die Unsicherheit hier durch $\increment T_{-} = \increment \kappa \cdot 2 \pi \sqrt{\frac{l}{g + 2\kappa }^2} $ berechnet. \\ \\

  \noindent Bei der gekoppelten Schwingung werden die Schwingungdauer und die Schwebungsdauer gemessen. Die Werte berechnen sich als Mittelwert von zehn Messwerten.
  Es ergibt sich
  \begin{equation*}
    \bar{T}_{\text{schwing}} = \SI{1.204(27)}{\second}
  \end{equation*} 
  und
  \begin{equation*}
    \bar{T}_{\text{schweb}} = \SI{8.50(14)}{\second}.
  \end{equation*}

  \noindent Der Theoriewert der Schwebungsdauer berechnet sich über die Gleichung \eqref{eqn:Ts} $T_{\text{s}} = \frac{T_{+} \cdot T_{-}}{T_{+} - T_{-}}$, es werden
  die Theoriewerte genutzt. Der Fehler ergibt durch den Fehler des Theoriewertes der gegensinnigen Schwingung. Nach der Gauss'schen Fehlerfortpflanzung \eqref{eqn:gauss}
  errechnet sich der Fehler durch 
  \begin{equation}\label{eqn:fehlerts}
    \increment T_{\text{s}} = \increment T_{\text{theo, -}} \cdot \left(\frac{T_{+}^2}{(T_{+} - T_{-})^2}\right).
  \end{equation}
  Somit lautet der Theoriewert der Schwebungsdauer
  \begin{equation*}
    T_ {\text{theo, s}} = \SI{80(4)}{\second}. 
  \end{equation*} \\ \\

  \noindent Die Kreisfrequenzen $\omega$ berechnen sich aus den jeweiligen Schwingungsdauern, es ergibt sich der Fehler nach der Gauss'schen Fehlerrechnung 
  \eqref{eqn:gauss}. Es gilt somit
  \begin{equation} \label{eqn:omega}
    \omega = \frac{2 \pi}{T} \pm \increment  T \cdot \frac{2 \pi}{T^2}. 
  \end{equation}
  Dementsprechend lauten die Werte: 
  \begin{align*}
    \omega_{+} &= \SI{4.985(22)}{\hertz} &         \omega_{\text{theo}, +} &= \SI{5.293}{\hertz}  \\
    \omega_{-} &= \SI{5.780(34)}{\hertz} &         \omega_{\text{theo}, -} &= \SI{5.372(4)}{\hertz} \\
    \omega_{\text{s}} &= \SI{0.739(12)}{\hertz} &  \omega_{\text{theo, s}} &= \SI{-0.079(4)}{\hertz} 
  \end{align*}
  Da die Kreisfrequenz von der Schwebung über $\omega_{+} - \omega_{-} $ \eqref{eqn:ws} berechnet wird, ist hier der Fehler durch die Formel 
  \begin{equation} \label{eqn:fehlerws}
    \increment \omega_{\text{s}} = \sqrt{ (\increment \omega_{+} )^2 + (\increment \omega_{-} )^2 }
  \end{equation}
  zu rechnen. 

\subsection{Lange Pendellänge}

  Es wird eine Pendellänge von $l = \SI{1.02}{\metre}$ genutzt. Es wurde analog zur kurzen Pendellänge für die einzelnen Pendel und die 
  gegensinnige Schwingung jeweils $5$ Schwingungsdauern gemessen, zwei Schwingungsdauern beim gekoppelten Pendel und eine Schwebungsdauer. 
  Die Messdaten werden in der \autoref{tab:102cm} eintragen. 

  \begin{table}
    \centering
    \caption{Die Messdaten von der Messung der Pendellänge $l = \SI{102}{\centi\metre}$.}
    \label{tab:102cm}
    \begin{tabular}{c c c c c}
      \toprule
      $ 5 \cdot T_{\text{rechts}} \mathbin{/} \si{\second} $ & $ 5 \cdot T_{\text{links}} \mathbin{/} \si{\second} $ & $ 5 \cdot T_{\text{gegen}} \mathbin{/} \si{\second} $ &
      $ 2 \cdot T_{\text{schwing}} \mathbin{/} \si{\second} $ & $ T_{\text{schweb}} \mathbin{/} \si{\second} $\\
      \midrule
      \multicolumn{5}{l}{Messdaten von Sonia}\\
      \midrule
      9.66  &   10.06 &  9.44 &   3.46 &   26.33  \\
      10.02 &   9.99  &  9.46 &   3.79 &   26.52  \\
      9.69  &   9.49  &  9.37 &   3.70 &   26.51  \\
      9.40  &   9.73  &  9.40 &   3.85 &   26.20  \\
      9.76  &   9.72  &  9.29 &   3.92 &   26.37  \\
      9.66  &   10.04 &  9.33 &        &          \\
      9.83  &   9.96  &  9.33 &        &          \\
      9.82  &   9.94  &  9.37 &        &          \\
      9.46  &   9.87  &  9.26 &        &          \\
      9.82  &   10.13 &  9.30 &        &          \\
      \midrule
      \multicolumn{5}{l}{Messdaten von Jana}\\
      \midrule
      9.58  &   10.03 &  8.96 &   3.86 &   25.72  \\
      9.86  &   9.88  &  9.14 &   3.55 &   27.57  \\
      9.92  &   9.90  &  9.03 &   3.35 &   26.58  \\
      9.65  &   9.88  &  9.02 &   3.41 &   25.26  \\
      9.89  &   9.92  &  9.16 &   3.54 &   26.10  \\
      9.68  &   9.69  &  9.10 &        &          \\
      9.51  &   9.60  &  9.04 &        &          \\
      9.80  &   9.83  &  9.31 &        &          \\
      9.92  &   9.75  &  9.42 &        &          \\
      9.93  &   9.76  &  9.11 &        &          \\
      \bottomrule
    \end{tabular}
  \end{table}

  \noindent Analog zur kurzen Pendellänge werden die Mittelwerte nach den Formeln \eqref{eqn:mittel} und \eqref{eqn:fehlermittel} ermittelt. 
  Für das rechte Pendel ergibt sich der Wert
  \begin{equation*}
    \bar{T}_{\text{rechts}} = \SI{1.949(7)}{\second}, 
  \end{equation*}
  für das linke Pendel eine von 
  \begin{equation*}
    \bar{T}_{\text{links}} = \SI{1.972(7)}{\second}. 
  \end{equation*}
  Damit ist die experimentell ermittelte Schwingungsdauer der gleichsinnigen Schwingung $ T_{+} = \SI{1.960(5)}{\second}$.
  
  \noindent Der Theoriewert der gleichsinnigen Schwingung berechnet sich über \eqref{eqn:Tplus} und lautet 
  \begin{equation*}
    T_{\text{theo, +}} = \SI{2.026}{\second}.
  \end{equation*} \\ \\

  \noindent Die Schwingungsdauer der gegensinnigen Schwingung ergibt sich zu 
  \begin{equation*}
      T_{-} = \bar{T}_{\text{gegen}} = \SI{1.848(7)}{\second}.
  \end{equation*}
  Somit kann nun die Kopplungskonstante $\kappa $ nach \eqref{eqn:kappa} bestimmt werden. Die Fehlerrechnung erfolgt analog zu kurzen Pendellänge
  nach \eqref{eqn:fehlerk}. 
  Die Kopplungskonstante $\kappa$ lautet 
  \begin{equation*}
      \kappa = \num{0.059(5)}.
  \end{equation*}

  \noindent Der Theoriewert der gegensinnigen Schwingung lässt sich nun zu 
  \begin{equation*}
      T_{\text{theo, -}} = \SI{2.0144(9)}{\second}, 
  \end{equation*}
  wobei sich der Fehler gemäß der Gauss'schen Fehlerfortpflanzung \eqref{eqn:gauss} über die Formel $\increment T_{-} = \increment \kappa \cdot 2 \pi \sqrt{\frac{l}{g + 2\kappa }^2} $ berechnet. \\ \\

  \noindent Für die gekoppelte Schwingung berechnen sich die Schwingungsdauer und die Schwebungsdauer über die Mittelwerte gemäß der Formel \eqref{eqn:mittel} und \eqref{eqn:fehlermittel}. 
  Die Schwingungsdauer ist 
  \begin{equation*}
    \bar{T}_{\text{schwing}} = \SI{1.821(31)}{\second}, 
  \end{equation*} 
  die Schwebungsdauer
  \begin{equation*}
    \bar{T}_{\text{schweb}} = \SI{26.32(18)}{\second}.
  \end{equation*}

  \noindent Der Theoriewert für die Schwebungsdauer wird mit der Formel \eqref{eqn:Ts} berechnet, der Fehler nach \eqref{eqn:fehlerts}.
  Es ergibt sich
  \begin{equation*}
      T_ {\text{theo, s}} = \SI{340(26)}{\second}. 
  \end{equation*} \\ \\

  \noindent Die Kreisfrequenzen berechnen sich nach der Formel \eqref{eqn:omega} aus den gemessenen Schwingungsdauern. 
  Die Werte lauten: 
  \begin{align*}
      \omega_{+} &= \SI{3.205(8)}{\hertz} &         \omega_{\text{theo}, +} &= \SI{3.100}{\hertz}  \\
      \omega_{-} &= \SI{3.399(13)}{\hertz} &         \omega_{\text{theo}, -} &= \SI{3.1192(14)}{\hertz} \\
      \omega_{\text{s}} &= \SI{0.2388(16)}{\hertz} &  \omega_{\text{theo, s}} &= \SI{-0.0185(14)}{\hertz} 
  \end{align*}
  Hierbei wird die Kreisfrequenz der Schwebung über die Formel \eqref{eqn:ws} gerechnet und der Fehler über die Gleichung \eqref{eqn:fehlerws}. 