\section{Auswertung}
\label{sec:Auswertung}
\subsection{Teil a)}
Im folgenden sind die Wurzeln der Photoströme gegen die Bremsspannung aufgetragen. 
Eine negative Bremsspannung steht hierbei für eine beschleunigende Spannung.

\subsubsection{Rot}
Die Werte der Messreihen für rotes Licht sind in der \autoref{tab:roteslicht} zu finden. 
\begin{figure}
  \centering
  \includegraphics{build/rot.pdf}
  \caption{Die Wurzel des Photostroms in Abhängigkeit der angelegten Bremsspannung bei der Bestrahlung mit rotem Licht.}
  \label{fig:rot}
\end{figure}
\noindent
In der \autoref{fig:rot} ist die Wurzel des Photostroms, welcher bei rotem Licht aufgetreten ist, gegen die angelegte Bremsspannung aufgetragen.
Mithilfe von Python wurde eine Ausgleichsrechnung der Form $\sqrt{I} = a \cdot U_{\text{Br}} + c $ gemacht.
Für die Ausgleichsrechnugn wurden nur die Werte bis zu einer Gegenspannung von $\SI{1}{\volt}$ betrachtet.
Die Werte ergeben sich zu:
\begin{align*}
  a_{\text{rot}} &= \SI{-3.4 \pm 0.1}{\volt\per\sqrt{\pico\ampere}}\\
  c_{\text{rot}} &= \SI{1.98 \pm 0.2}{\sqrt{\pico\ampere}}
\end{align*}
Der Schnittpunkt der Ausgleichsgeraden mit der x-Achse beschreibt den Wert der Grenzspannung und wird berechnet durch:
\begin{equation}\label{eqn:U_G}
  U_{\text{g}}=-\frac{c}{a}
\end{equation}
Für die Bestrahlung mit rotem Licht lautet die Grenzspannung $U_{\text{g}}$:
\begin{equation*}
  U_{\text{g},\text{rot}} = \SI{0.58\pm0.06}{\volt}
\end{equation*}

\subsubsection{Grün}
Bei der Untersuchung des Photoeffektes mit grünem Licht ist bei der Messung ein plötzlicher Einbruch des gemessenen Photostroms aufgefallen.
Dieser ist in der \autoref{tab:gruneslicht} der Messwerte durch eine weitere Mittellinie gekennzeichnet.
In dem Plot \autoref{fig:grun} sind die verschiedenen Messungen durch verschiedene Farben markiert.
\begin{figure}
  \centering
  \includegraphics{build/grun.pdf}
  \caption{Die Wurzel des Photostroms in Abhängigkeit der angelegten Bremsspannung bei der Bestrahlung mit grünem Licht.}
  \label{fig:grun}
\end{figure}
\noindent
Für die Ausgleichsrechnungen der Form $\sqrt{I} = a \cdot U_{\text{Br}} + c $ berechneten sich folgende Werte:
\begin{align*}
  a_{\text{grün},1} &= \SI{-15.1 \pm 0.6}{\volt\per\sqrt{\pico\ampere}}\\
  c_{\text{grün},1} &= \SI{29.3 \pm 1.4}{\sqrt{\pico\ampere}}\\
  \intertext{und}\\
  a_{\text{grün},2} &= \SI{-15.8 \pm 0.9}{\volt\per\sqrt{\pico\ampere}}\\
  c_{\text{grün},2} &= \SI{0.2 \pm 0.4}{\sqrt{\pico\ampere}}
\end{align*}
Hier wurde für die Rechnung alle der schwarz markierten Messwerte ab $\SI{-4}{\volt}$ genommen, bei den blau markierten Messwerten wurden in der Ausgleichsrechung nur die Werte bis zu einer Gegenspannung von $\SI{0.25}{\volt}$ berücksichtigt.
Die jeweiligen Grenzspannungen $U_{\text{g}}$ berechnen sich nach \eqref{eqn:U_G} zu:
\begin{align*}
  U_{\text{g},\text{grün},1} &= \SI{1.94 \pm 0.12}{\volt}\\
  U_{\text{g},\text{grün},2} &= \SI{0.014 \pm 0.028}{\volt}\\
\end{align*}
Hier wurde von den beiden Nullstellen der Mittelwert gebildet, mit welchem dann auch im weiteren gerechnet wurde.
\begin{equation*}
  U_{\text{g},\text{grün}} = \SI{0.97 \pm 0.06}{\volt}
\end{equation*}

\subsubsection{Violett}
Es wurde der Messvorgang mit violettem Licht wiederholt.
\begin{figure}
  \includegraphics{build/violett.pdf}
  \caption{Die Wurzel des Photostroms in Abhängigkeit der angelegten Bremsspannung bei der Bestrahlung mit violettem Licht.}
  \label{fig:violett}
\end{figure}
\noindent
In der \autoref{fig:violett} ist neben den Messwerten aus \autoref{tab:violicht}, wobei von dem Strom die Wurzel $\sqrt{I}$ aufgetragen wurde, 
eine Ausgleichsgerade der Form $\sqrt{I} = a \cdot U_{\text{Br}} + c $ abgebildet.
Hierfür wurden die Messwerte bis zu einer Gegenspannung von $\SI{1}{\volt}$ betrachtet.
Die Parameter wurden von Python zu
\begin{align*}
  a_{\text{violett}} &= \SI{-19.6\pm 0.9}{\volt\per\sqrt{\pico\ampere}}\\
  c_{\text{violett}} &= \SI{22.8 \pm 2.6}{\sqrt{\pico\ampere}}
\end{align*}
berechnet.
Damit lässt sich über \eqref{eqn:U_G} die Grenzspannung ermitteln:
\begin{equation*}
  U_{\text{g},\text{violett}} = \SI{1.17 \pm 0.14}{\volt}
\end{equation*}

\subsubsection{Gelb}
Für die Bestrahlung mit gelben Licht wurde eine größere Messreihe genommen, wie in \autoref{tab:gelblicht} zu sehen ist.
\begin{figure}[H]
  \includegraphics{build/gelb_a.pdf}
  \caption{Die Wurzel des Photostroms in Abhängigkeit der angelegten Bremsspannung bei der Bestrahlung mit gelben Licht.}
  \label{fig:gelba}
\end{figure}
\noindent
Wie in der \autoref{fig:gelba} zu sehen ist, bildet sich bei einer so langen Messreihe Richtung $-\infty$ eine Abflachung der Messwerte.
Da diese Messwerte den linearen Zusammenhang nicht wiederspiegeln, werden sie in der Ausgleichsrechnung der Form $\sqrt{I} = a \cdot U_{\text{Br}} + c $ nicht beachtet.
Die Messwerte von $\SI{-10}{\volt}$ bis $\SI{-0.01}{\volt}$ wurden in der Ausgleichsrechnung berücksichtigt.
Es wurden die Werte zu 
\begin{align*}
  a_{\text{gelb}} &= \SI{-13.3 \pm 0.9}{\volt\per\sqrt{\pico\ampere}}\\
  c_{\text{gelb}} &= \SI{3.4 \pm 1.9}{\sqrt{\pico\ampere}}
\end{align*}
ermittelt, sodass sich für die Grenzspannung der folgende Wert ergibt:
\begin{equation*}
  U_{\text{g},\text{gelb}} = \SI{0.25 \pm 0.15}{\volt}
\end{equation*}

\subsection{Teil b)}
In diesem Aufgabenteil werden die in Teil a) ermittelten Werte gegen die Frequenzen des jeweilig farbigen Lichtes in einem Diagramm aufgetragen. 
Die Frequenzen und Wellenlängen der Farben lassen sich in der \autoref{tab:nu} finden. 
\begin{table}
  \centering
  \caption{Die Wellenlängen und Frequenzen der unterschiedlichen Farben.}
  \label{tab:nu}
  \begin{tabular}{S[table-format=7] S[table-format=3] S[table-format=1.2]}
    \toprule
    & $\lambda [\si{\nano\metre}]$ & $\nu [\SI{e14}{\hertz}]$\\
    \midrule
    \text{rot}    & 615 & 4.87\\
    \text{grün}   & 546 & 5.49\\
    \text{violett} & 435 & 6.89\\
    \text{gelb} & 578 & 5.18 \\
    \bottomrule
  \end{tabular}
\end{table}

\begin{figure}[H]
  \centering
  \includegraphics[width=\textwidth]{build/b.pdf}
  \caption{Die Grenzspannung aufgetragen gegen die Frequenzen des jeweilig farbigen Lichtes.}
  \label{fig:b}
\end{figure}
\noindent
In der \autoref{fig:b} sind diese Punkte abgebildet. 
Zusätzlich wurde eine lineare Ausgleichsrechnung durchgeführt, welche die Form $U_{\text{g}} = a\cdot \nu - c$ hat.
Für die Parameter ergeben sich die Werte: 
\begin{align*}
  a_{\text{b}} &= \SI{3.5(21)e-15}{\volt\second}\\
  c_{\text{b}} &= \SI{-1.2 \pm 1.2}{\volt}
\end{align*}
Nach Vergleich mit der Formel \eqref{eqn:furb} folgen die Zusammenhänge
\begin{align*}
  \frac{\symup{h}}{\symup{e}_0} &= a_{\text{b}}\\
  \frac{A_{\text{k}}}{\symup{e}_0} &= - c_{\text{b}},
\end{align*}
sodass für die im Experiment gemessenen Werte gilt:
\begin{align*}
  \frac{\symup{h}}{\symup{e}_0} &= \SI{3.5(21)e-15}{\volt\second}\\
  A_{\text{k}} &= \SI{1.2 \pm 1.2}{\electronvolt}
\end{align*}

\subsection{Teil c)}
Nun wird die Messreihe mit gelbem Licht der Wellenlange $\lambda = \SI{578}{\nano\metre}$ nochmal genauer betrachtet.
Dazu wurden diesmal die gemessenen Ströme $I$ in Abhängigkeit der angelegten Bremsspannung $U_{\text{Br}}$ in dem Diagramm \autoref{fig:gelbc} abgebildet.
\begin{figure}[H]
  \centering
  \includegraphics[width=\textwidth]{build/gelb_c.pdf}
  \caption{Die Messwerte aus \autoref{tab:gelblicht} aufgetragen in einem $U$-$I$- Diagramm.}
  \label{fig:gelbc}
\end{figure}
\noindent
In der \autoref{fig:gelbc} ist bei einer hohen Beschleunigungsspannung (also negativer Bremsspannung) eine Annäherung an einen Sättigungswert zu sehen.
Dies bildet keinen Widerspruch zum Ohm'schen Gesetz, da die Anzahl der ausgelösten Elektronen, also der gemessene Strom von der Intensität des Lichtes abhängig ist und nicht von der Höhe der Spannung.
Die Intensität des Lichtes wurde im Experiment konstant gehalten.
Da die Energie der Elektronen in der Photokathode unterschiedlich sein kann, gibt es immer Elektronen, die nur durch sehr hohe angelegt Beschleunigungsspannungen gemessen werden können.
Jedoch wird ab einer gewissen Spannung fast jedes ausgelöste Elektron gemessen.
Bei einer Photozelle, welche alle aus der Kathode ausgelösten Elektronen wahrnimmt,  kann der Photostrom bereits bei endlichen Beschleunigungsspannungen seinen Sättigungswert erreichen.\\
\\
Die Elektronen in der Photokathode haben bereits verschiedene kinetische Energien, welche ,wie in der Theorie \ref{sec:Theorie} beschrieben, durch die Fermi-Dirac-Verteilung gegeben ist.
Werden sie aus der Kathode ausgelöst, so haben die gelösten Elektronen dann auch verschieden hohe Energien.
Somit können bei bestimmten Gegenspannungen oder auch Beschleunignungsspannungen nicht alle ausgelösten Elektronen die Anode erreichen.
Daher beginnt der Photostrom schon vor Erreichen der Grenzspannung $U_{\text{g}}$ zu sinken. \\
\\
Die Photokathode besteht aus einem Material, welches bei Raumtemperatur verdampft.
Daher werden auch Elektronen aus der Kathode ausgelöst, welche sich an der Anode lagern.
Ist die Gegenspannung  groß genug, werden die Elektronen von der Anode zurück zur Kathode beschleunigt.
Daher werden bei postiven Bremsspannungen teilweise negative Ströme beobachtet.
Da es sich hierbei um deutlich weniger Elektronen handelt, als bei den vom Photoeffekt ausgelösten Elektronen, stellt sich hier deutlich schneller ein Sättigungswert ein.\\
Die negativen Ströme werden im Experiment schon bei Licht der Wellenlänge $\lambda = \SI{650}{\nano\metre}$ beobachtet, also recht niederenergetischem Licht.
Somit kann die Annahme  getroffen werden, dass die Austrittarbeit der Kathode $A_{\text{k}}$ auch recht klein ist.

\subsection{Messwerte}
\begin{table}
  \centering
  \caption{Die Werte von der Messung mit rotem Licht.}
  \label{tab:roteslicht}
  \begin{tabular}{S[table-format=2.3] S[table-format=3.1]}
    \toprule
    $U_{\text{Br}} [\si{\volt}]$ & $I_{\text{photo}} [\si{\pico\ampere}]$\\
    \midrule
    -5.0   &  300\\   
    -4.5   &  275\\   
    -4.0   &  235\\   
    -3.5   &  205\\   
    -3.0   &  175\\   
    -2.5   &  160\\   
    -2.0   &  100\\   
    -1.75  &  75 \\   
    -1.5   &  56 \\   
    -1.4   &  52 \\   
    -1.3   &  43 \\   
    -1.2   &  36 \\   
    -1.1   &  30 \\   
    -1.0   &  21 \\   
    -0.9   &  21 \\   
    -0.8   &  18 \\   
    -0.7   &  14 \\   
    -0.6   &  10 \\   
    -0.5   &  14 \\   
    -0.4   &  11 \\   
    -0.3   &  8  \\   
    -0.2   &  6  \\   
    -0.1   &  4  \\   
    0.001  &  4  \\   
    0.1    &  2  \\   
    0.25   &  1  \\   
    0.5    &  0.5\\   
    1.0    &  0  \\   
    2.0    &  0  \\   
    3.0    &  0  \\   
    4.0    &  0  \\   
    5.0    &  0  \\
    \bottomrule
  \end{tabular}
\end{table}

\begin{table}
  \centering
  \caption{Die Werte von der Messung mit grünem Licht. Die Werte unterhalb der zweiten Linie sind nach einer Unterbrechung gemessen wurden.}
  \label{tab:gruneslicht}
  \begin{tabular}{S[table-format=2.3] S[table-format=4]}
    \toprule
    $U_{\text{Br}} [\si{\volt}]$ & $I_{\text{photo}} [\si{\pico\ampere}]$\\
    \midrule
    -5     & 8800 \\   
    -4.5   & 8200 \\   
    -4.0   & 7600 \\   
    -3.5   & 6100 \\   
    -3.0   & 5500 \\   
    -2.75  & 5000 \\   
    -2.5   & 4700 \\   
    -2.25  & 4300 \\   
    -2.0   & 3900 \\   
    -1.75  & 3500 \\   
    -1.5   & 3100 \\   
    -1.25  & 2600 \\   
    -1.0   & 2100 \\   
    -0.75  & 1500 \\   
    -0.5   & 1200 \\   
    -0.25  & 950  \\   
    -0.15  & 760  \\   
    -0.1   & 750  \\ 
    \midrule
    -1     &    300 \\
    -0.75  &    125 \\
    -0.5   &    50  \\
    -0.25  &    14  \\
    -0.15  &    5   \\
    -0.1   &    4   \\
    -0.05  &    2   \\
    0      &    1   \\
    0.25   &    0   \\
    0.5    &    0   \\
    1      &    0   \\
    3      &    0   \\
    5      &    0   \\ 
    \bottomrule
  \end{tabular}
\end{table}


\begin{table}
  \centering
  \caption{Die Werte von der Messung mit violettem Licht.}
  \label{tab:violicht}
  \begin{tabular}{S[table-format=2.2] S[table-format=4]}
    \toprule
    $U_{\text{Br}} [\si{\volt}]$ & $I_{\text{photo}} [\si{\pico\ampere}]$\\
    \midrule
    -5     &   13000 \\
    -4.5   &   11500 \\
    -4     &   10000 \\
    -3.5   &   8500  \\
    -3     &   7100  \\
    -2.5   &   5400  \\
    -2     &   4600  \\
    -1.5   &   3600  \\
    -1     &   2350  \\
    -0.5   &   1500  \\    
    0      &   400   \\
    0.01   &   180   \\
    0.5    &   23    \\
    1      &   0     \\
    3      &   0     \\
    5      &   0     \\
    \bottomrule
  \end{tabular}
\end{table}


\begin{table}
  \centering
  \caption{Die Werte von der Messung mit gelben Licht.}
  \label{tab:gelblicht}
  \begin{tabular}{S[table-format=2.2] S[table-format=4] S[table-format=2.2] S[table-format=4]}
    \toprule
    $U_{\text{Br}} [\si{\volt}]$ & $I_{\text{photo}} [\si{\pico\ampere}]$ & $U_{\text{Br}} [\si{\volt}]$ & $I_{\text{photo}} [\si{\pico\ampere}]$\\
    \midrule
    -19     &   8300  & -0.95   &    150\\
    -18     &   7800  & -0.9    &    120\\
    -16     &   7500  & -0.85   &    120\\
    -13     &   7200  & -0.7    &    100\\
    -12     &   7000  & -0.65   &     50\\ 
    -11     &   6800  & -0.6    &     30\\ 
    -10.5   &   6600  & -0.55   &     20\\  
    -10     &   6500  & -0.5    &     15\\
    -9      &   6000  & -0.4    &     10\\
    -8      &   5800  & -0.35   &      1\\  
    -7      &   5400  & -0.3    &      1\\
    -6.5    &   5000  & -0.25   &    0.5\\ 
    -5      &   4400  & -0.15   &   0.25\\
    -4.5    &   4000  & -0.1    &  0.125\\
    -4      &   3400  & -0.05   &    0.1\\  
    -3.5    &   3000  & -0.01   &      0\\
    -3      &   2750  &  0.01   &      0\\ 
    -2.5    &   2000  &  0.5    &      0\\
    -2      &   1750  &  1      &      0\\
    -1.5    &   1300  &  2      &      0\\ 
    -1.4    &   1250  &  3      &      0\\  
    -1.3    &   1000  &  4      &      0\\
    -1.25   &    700  &  5      &      0\\
    -1.2    &    500  &  10     &      0\\  
    -1.15   &    400  &  15     &      0\\
    -1.1    &    600  &  19     &      0\\
    -1      &    600  &         &       \\
    \bottomrule
  \end{tabular}
\end{table}