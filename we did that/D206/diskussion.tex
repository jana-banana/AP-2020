\section{Diskussion}
\label{sec:Diskussion}

Bei der Berechnung der Güteziffer fiel bereits auf, der reale Wert stark von der idealen Güteziffer abweicht.
Dies beruht wie oben genannt auf den Annahmen, die zur Formel der idealen Güteziffer führten.
Der Kompressor kann nicht adiabatisch arbeiten und auch die Reversiblität des Vorganges ist realitätsfern.
Zudem haben wir noch die Einflüsse der Umgebung außer Acht gelassen.
Die Beschränkung auf vier Stellen, an denen wir z.B. die Ableitung der Temperatur bestimmt haben, lässt die Möglichkeit offen, dass lokale Unregelmäßigkeiten die Messergebnisse beeinflussen.

Weitere mögliche Fehlerquellen wie schlecht abzulesende Anzeigen können wir nicht in unsere Diskussion einfließen lassen, da wir den Versuch nur anhand der Daten ausgewertet haben und ihn nicht durchführen konnten.
