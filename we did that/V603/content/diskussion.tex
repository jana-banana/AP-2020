\section{Diskussion}
\label{sec:Diskussion}

Die aus dem Emissionsspektrum ermittelten Daten für die Energien des $K_{\alpha}$- und $K_{\beta}$- Linien sind in der \autoref{tab:vergleich_energie} zu finden. 

\begin{table}[H]
    \centering
    \caption{Die Daten der Energien im Vergleich mit den Literaturwerten.}
    \label{tab:vergleich_energie}
    \begin{tabular}{c c @{${} \pm {}$} c S[table-format=4.1] S[table-format=2.1]}
        \toprule
        Linie & \multicolumn{2}{c}{$E_{K, \text{exp}} \, / \si{\electronvolt}$ } & {$E_{K, \text{lit}} \, / \si{\electronvolt} $ \cite{theoE}} & {$\increment E  \, / \si{\percent}$} \\
        \midrule
        $\alpha$ &  8043 & 34 & 8048.1 & 0.1 \\
        $\beta $  &  8910 & 40 & 8906.9 & -0.1 \\
        \bottomrule
    \end{tabular}
\end{table}

\noindent Es ist auffällig, dass die Theoriewerte sehr genau getroffen werden. Da die Aufnahme der Daten über einen Röntgenapparat passiert, welcher auch die 
Winkel des LiF-Kristalls eingestellt, ist von kleinen systematischen Fehlern auszugehen. Außerdem wird in Winkelabschnitten von $\increment \theta = \SI{0.1}{\degree}$ 
gemessen, sodass die Bestimmung des $K_{\alpha}$ und $K_{\beta}$ Winkels recht genau ist. \\


\noindent Die experimentell ermittelte Comptonwellenlänge $\lambda_{\text{c, exp}} = \SI{3.8(11)}{\pico\metre}$ zeigt eine Abweichung von $ \SI{54.9}{\percent}$
zur theoretischen Compton-Wellenlänge $\lambda_{\text{c, lit}} = \frac{\symup{h}}{\symup{m}_\text{e} \cdot \symup{c}} = \SI{2.42}{\pico\metre}$. Diese Abweichung ist vielleicht damit erklärbar, 
dass die Wellenlängen zu den jeweiligen Transmissionen mit den Werten der Ausgleichsrechnung berechnet werden und sich somit die Fehler fortpflanzen. 
Es wird keine Totzeitkorrektur für die Impulsdaten gemacht, da aufgrund der geringen Anzahl bei einer recht hohen Integrationszeit wahrscheinlich sehr wenige
Impulse nicht vom Geiger-Müller-Zählrohr registriert werden. \\
Der Compton-Effekt tritt nicht im sichtbaren Spektrum auf, da die Zunahme der Wellenlänge relativ zu ihr sehr gering ist. Dadurch scheint die Streuung ohne
Energieverlust abzulaufen und es ist kein Compton-Effekt nachweisbar. 