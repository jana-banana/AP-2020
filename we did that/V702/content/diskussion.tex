\section{Diskussion}
\label{sec:Diskussion}
\begin{table}
    \centering
    \caption{Vergleich der errechneten Halbwertszeiten mit denen aus der Literatur.}
    \label{tab:diskussion}
    \begin{tabular}{S[table-format=10]
                    S[table-format=3.1]
                    S[table-format=3.1]
                    S[table-format=6]}
    \toprule
    &{$T_{\text{H},\text{theo}} [\si{\second}]$}&{$T_{\text{H},\text{exp}} [\si{\second}]$}&{$\increment T$}\\
    \midrule
    $\text{Vanadium}$ &         &       &   \\
    \midrule
    $\text{komplett}$ & 224.6 $\:$ \cite{vana-lit} & 219 & $\SI{2.5}{\percent}$\\
    $\text{Auswahl}$ &      & 180 & $\SI{19.85}{\percent}$\\
    \midrule
    $\text{Rhodium}$ &                       &         &   \\
    \midrule
    $\text{langlebig}$ & 260 $\:$\cite{rho-lang} & 212 & $\SI{18.46}{\percent}$\\
    $\text{kurzlebig}$ & 43.2 $\:$\cite{rho-kurz} & 57.3 &  $\SI{24.67}{\percent}$\\
    \bottomrule
    \end{tabular}
\end{table}
In der \autoref{tab:diskussion} sind die im Experiment ermittelten Werte für die jeweiligen Halbwertszeiten, die Literaturwerte der Halbwertszeiten und die prozentuale Abweichung dieser Werte angegeben. 
Bei der Vanadiumprobe wurde auf zwei verschiedene Arten die Halbwertszeit ermittelt werden, wobei die zweite Methode auf einen genaueren Wert führen sollte.
Dies ist in dem Experiment nicht der Fall gewesen.
Es ist davon auszugehen, dass die Halbwertszeit aus der ersten, ungenaueren, Bestimmung nur zufällig näher am Literaturwert liegt.\\
Bei der Rhodiumprobe weichen beide Werte ungefähr $\SI{20}{\percent}$ von ihrem Literaturwert ab. 
Hier die größte zusätzliche Fehlerquelle die ungenaue Bestimmung  von $t^*$. 
Der Zeitpunkt, ab dem der langlebige Zerfall überwiegt, ist in der \autoref{fig:rhodium} nicht klar sichtbar und kann nur vermutet werden.
