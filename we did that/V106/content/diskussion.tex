\section{Diskussion}
\label{sec:Diskussion}

In der \autoref{tab:vergleich} sind die ermittelten Werte neben den theoretisch errechnenten Werten zu sehen. 

\begin{table}
    \centering
    \caption{Der Vergleich der experimentell ermittelten und theoretisch errechneten Werte.}
    \label{tab:vergleich}
    \begin{tabular}{c c c c c c c}
        \toprule
        &\multicolumn{2}{c}{$T \mathbin{/} \si{\s}$} & & \multicolumn{2}{c}{$ \omega \mathbin{/} \si{\hertz}$}\\
        \cmidrule(lr){2-3} \cmidrule(lr){5-6}
        Art & exp & theo & $\increment T \mathbin{/}$ \% & exp & theo & $\increment \omega \mathbin{/}$ \% \\
        \midrule
        \multicolumn{7}{l}{Kurze Pendellänge ($l = \SI{0.35}{\metre}$)}\\
        \midrule
        gleichsinnig &  1.260 $\pm$ 0.005 &     1.187 &     6.2     &   4.985 $\pm$ 0.022 &    5.293 &      5.8 \\
        gegensinnig  &  1.087 $\pm$ 0.006 &     1.169 &     7.1     &   5.780 $\pm$ 0.034 &    5.372 &      7.6 \\
        Schwebung    &  8.500 $\pm$ 0.140 &    80     &    89.3     &   0.739 $\pm$ 0.012 &   -0.079 &      1040 \\
        \midrule
        \multicolumn{7}{l}{Lange Pendellänge ($l = \SI{1.02}{\metre}$)}\\
        \midrule
        gleichsinnig &  1.969 $\pm$ 0.005 &     2.026 &     3.2     &   3.205 $\pm$ 0.008 &    3.100 &      3.4 \\
        gegensinnig  &  1.848 $\pm$ 0.007 &     2.014 &     8.2     &   3.399 $\pm$ 0.013 &    3.119 &      9.0 \\
        Schwebung    & 26.320 $\pm$ 0.180 &   340     &    92.3     &   0.238 $\pm$ 0.001 &   -0.018 &      1390\\
        \bottomrule
    \end{tabular}
\end{table}

\noindent Es ist zu sehen, dass die Abweichung für die gleichsinnige Schwingung sowohl bei der kurzen, als auch bei der langen Pendellänge recht klein ist, das
gemessene Ergebnis liegt somit sehr nah am Theoriewert. Die Werte für die gegensinnige Schwingung haben eine größere Abweichung, welche jedoch unter $\SI{10}{\percent}$
liegt. Dies ist damit zu erklären, dass die Theoriewerte mit der experimentell ermittelten Kopplungskonstante berechnet werden, wodurch sich die Fehler fortpflanzen 
können. Außerdem wurde der Mittelwert über weniger Werte gebildet, wodurch einzelne Fehler einen größeren Einfluss haben.
Die Schwebungsdauer hat sehr große Abweichungen. Dies kann einerseits daran liegen, dass nur eine Schwebungsdauer gemessen wird, andererseits wird der Mittelwert
auch über weniger Werte gebildet. Die Messung der Schwebungsdauer ist auch die komplizierteste, da es einen größeren Spielraum gibt, in dem eine Schwebungsdauer
für einen Zeitmesser durch sein kann.  
Analog kann man das Beschriebene bei den Frequenzen sehen, da die Werte hier besonders für die Schwebung sehr klein sind, ist die
prozentualen Abweichungen nocheinmal deutlich höher. Zwischen den verschiedenen Pendellängen lässt sich keine andere Entwicklung feststellen, die Unterschiede haben
ein normales Maß. \\ \\

\noindent Das Experiment hat relativ viele Fehlerquellen. Erstmal werden die Längen der Pendel mit einem Maßband gemessen, der Mittelpunkt der Masse und
der Mittelpunkt der Aufhägung sind nicht klar zu sehen, sondern müssen erahnt werden. Für die ganzen Rechnungen wird die Kleinwinkelnäherung angenommen, jedoch ist in dem 
Versuchsaufbau kein Winkelmesser dabei, sodass dies nicht genau überprüft werden kann. Selbstverständlich kommt es durch Reibung an der Aufhängung und 
Luftwiderstand zu Energieverlusten, welche theoretisch auch nicht einberechnet werden. Durch das Messen von zwei Leuten gleichzeitig werden Abweichungen, die durch
die Perspektive und die Schnelligkeit des Zeitmessers kommen, minimiert. Durch das Aufnehmen von mehr Werten kann die Messung verbessert werden, jedoch verbessert
die Messung über mehr Schwingungsperioden das Ergebnis nicht. 