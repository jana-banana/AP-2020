\section{Diskussion}
\label{sec:Diskussion}
Die Fehler der Messwerte sind größtenteils nur aus den Toleranzen der Bauteile errechnet und liegen dort im angemessenen Bereich.
Eine Verbesserung ist nur über neue Bauteile mit geringeren Toleranzen möglich.
Da wir nicht so viele Referenzbauteile mit bekanntem Wert zur Verfügung hatten, konnten wir nur wenige Messungen derselben zu bestimmenden Größe machen, teilweise sogar nur eine.
Daher sind die Fehler der Mittelwerte teilweise recht groß, wie bei der Kapazität in \ref{sec:ausw_kap}.\\
Die Induktivitäten sollten eigentlich mit der Induktivitätsbrücke und der Maxwellbrücke ausgewertet werden.
Jedoch hat die Maxwellbrücke auch nach wiederholtem Aufbau nicht funktioniert.
Somit haben wir nur eine Messung zur Bestimmung durchführen können.\\
\\
Die Wien-Robinson-Brücke konnten wir nur mit zwei ähnlichen Kondensatoren aufbauen, da wir nur diese zur Verfügung hatten.
Die entnommenen Messdaten liegen doch sehr gut auf der Theoriekurve, doch für hohe Frequenzen weichen die Messdaten ab.\\
Der Klirrfaktor liegt auch in einem sehr gutem Bereich, was sich auch schon anhand des Diagramms \ref{fig:auswplot} vermuten lies.