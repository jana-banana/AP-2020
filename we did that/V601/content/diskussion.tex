\section{Diskussion}
\label{sec:Diskussion}

Die Versuchsdurchführung wurde durch den Temperaturregler und die Justierung des XY-Schreibers erschwert.
Der Temperaturregler funktionierte nicht einwandfrei.
Es mussten viel größere Temperaturerhöhungen vorgenommen werden um zufriedenstellende Ergebnisse zu erzielen.
Zudem sank dei eingestellte Temperatur schnell.
Aus diesem Grund wurde zunächst eine Franck-Hertz-Kurve bei $T = \SI{200}{\celsius}$ aufgenommen, dann gewartet bis
die Temperatur im zu untersuchenden Bereich einen Wert um $\SI{170}{\celsius}$ erreicht.
Somit wurde eine weitere Franck-Hertz-Kurve bei $T = \SI{176}{\celsius}$ aufgenommen. \\
\noindent
Die Justierung des XY-Schreibers erfolgte unteranderem durch Ausprobieren und verlief größstenteils gut.
Bei der zweiten Kurve aus dem ersten Teil des Versuches konnte zum Beispiel nicht vollständig in y-Achse des Blatts ausgenutzt werden. \\


\noindent
Die integrale und differentielle Energieverteilung zeigen gute Ergebnisse.
Ein Abflachen der Kurve ist beobachtbar.
Auch der Wendepunkt der integralen Energieverteilung konnte mithilfe der differentiellen Energieverteilung sehr gut bestimmt werden.
Es ergibt sich für das Konatktpotential $K = \SI{1.25}{\volt}$. \\


\noindent
Da die Franck-Hertz-Kurve zwei Mal aufgenommen wurde, wurde die Anregungsenergie und die Wellenlänge des emitttierten Photons jeweils 
zwei Mal bestimmt.
Folgende Ergebnisse wurden ermittelt
\begin{align*}
    T = \SI{200}{\celsius}&                  &E_{01}=\SI{5.0275}{\electronvolt} \, , \\
                          &          &\lambda_{200} = \SI{2.4661e-7}{\metre} \, , \\
                                    \\
    T = \SI{176}{\celsius}&                  &E_{01}=\SI{5.0962}{\electronvolt}  \, , \\
                          &          &\lambda_{176} = \SI{2.4329e-7}{\metre} \, . \\
\end{align*}
\noindent
Da der Theoriewert für die Anregungsenergie des Hg-Atoms bei $\SI{4.9}{\electronvolt}$ \cite{theo}, besteht bei $T = \SI{200}{\celsius}$ eine Abweichung von 2,6\%.
Bei $T = \SI{176}{\celsius}$ kommt es zu einer Abweichung von 4\%.
Die ermittelten Ergebnisse sind somit zufriedenstellend.\\
\noindent
Neben den oben genannten Bedingungen sind Ablesefehler eine weitere Fehlerquelle, die nicht ausgeschlossen werden kann.
Dennoch ist im Allgemeinen die Abweichung gering und die Ergebnisse gut.

% DER THEORIEWERT LIEGT BEI 4,9 eV sieht gut aus 
% https://lp.uni-goettingen.de/get/text/1612


%Temperaturregel ist scheiße, er macht nicht, was wir wollen
% E_{01}=\SI{5.0275}{\electronvolt}
% \lambda_{200} = \SI{2.4661e-07}{\metre}

% E_{01}=\SI{5.0962}{\electronvolt}
% \lambda_{176} = \SI{2.4329e-07}{\metre}