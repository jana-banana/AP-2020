\section{Zielsetzung}
\label{sec:Zielsetzung}
%In diesem Versuch wollen wir die Funktionsweise einer Wärmepumpe, die Wärme von einem kälteren in ein wärmeres Reservoir transportiert, genauer betrachten.
%Außerdem bestimmen wir anschließend Größen wie die Güteziffer und den Massendurchsatz der Wärmepumpe. %I don't even know 
\section{Theorie}
\label{sec:Theorie}
In diesem Versuch betrachten wir den Wärmetransport von einem kälteren in ein wärmeres Reservoir. 
Dafür benötigt man zusätzliche mechanische Arbeit.
\subsection{Die Güteziffer}
Die Güteziffer beschreibt das Verhältnis zwischen der übertragenen Wärme $Q_1$ und der dafür aufgebrachten Arbeit $A$.
Die übertragene Wärmemenge $Q_1$ setzt dich folgendermaßen aus $A$ und der dem kälteren Reservoir entnommene Wärmemenge $Q_2$ zusammen.
\begin{equation*}
Q_1 = Q_2 + A 
\end{equation*}
Mit der Grundannahme, dass es keine Temperaturänderung der Reservoire gibt, folgt aus dem zweiten Hauptsatz der Thermodynamik im irreversiblen (realistischen) Fall:
\begin{equation}
\frac{Q_1}{T_1 - \frac{Q_2}{T_2} > 0
\end{equation}
, wobei $T_1$ und $T_2$ die Wassertemperaturen der Reservoire bezeichnet.
Daraus folgt für die Güteziffer in idealen Bedingungen:
\begin{equation}
\nu_{\text{id}} = \frac{Q_1}{A} = \frac{T_1}{T_1 - T_2}
\end{equation}
Dementsprechent ist der Arbeitsaufwand für die Pumpe geringer, je kleiner die Temperaturdifferenz $\Delta T = |T_1 - T_2|$ ist.

Da in der Auswertung die Temperaturkurven mit Funktionen angenähert werden, benutzen wir statt der Differenzenquotienten im folgenden Differentialquotienten.


\cite{}
