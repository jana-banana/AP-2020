\section{Auswertung}
\label{sec:Auswertung}

\subsection{Messdaten und Diagramm}
Die Messwerte sind in einer Tabelle aufgetragen. Dabei sind die Temperaturen einmal in $\si{\celsius}$ und einmal in $\si{\kelvin}$ angegeben, 
den Drücken ist nach Anleitung \ref{anleitung} ein bar dazuaddiert worden.
\begin{table}
  \begin{tabular}{
    S[table-format=2.1] %Zeit
    S[table-format=2.1] %tempc
    S[table-format=3.2] %tempK
    S[table-format=2.2] %p_1
    S[table-format=2.1] %tempC
    S[table-format=3.2] %tempK
    S[table-format=1.1] %p_2
    S[table-format=3.0] %N
  }
  \toprule
  {$ t \mathbin{/} \si{\minute} $} &
  {$ T_1 \mathbin{/} \si{\celsius} $} &
  {$ T_2 \mathbin{/} \si{\kelvin}$} &
  {$ p_1 \mathbin{/} \si{\bar}$} &
  {$ T_2 \mathbin{/} \si{\celsius}$} &
  {$ T_2 \mathbin{/} \si{\kelvin}$} &
  {$ p_2 \mathbin{/} \si{\bar}$} &
  {$N_{\text{mech}} \mathbin{/} \si{\watt}$} \\
  \midrule
  \input{hope.txt}
  \end{tabular}
\end{table}

Dann haben wir die Temperaturen in $\si{\kelvin}$ gegen die Zeit in einem Diagramm aufgetragen. 
Die Verläufe haben wir mit einem Polynom 2. Grades ($T(t) = a\cdot t² + b\cdot t + c$)angenähert.
\begin{figure}
  \centering
  \includegraphics{tempK.pdf}
  \caption{Die gemessenen Temperaturen in Zeitverlauf mit den angenäherten Funktionen.}
  \label{fig:tempK}
\end{figure}
Für die Ausgleichsrechnung ergeben sich für die Konstanten $a, b$ und $c$:
\begin{align*}
  \intertext{Für } T_1(t)&=T(t):& \intertext{Für } T_2(t) &= T(t):\\
  a&=\SI{-0.0116}{\kelvin\per\second\squared}  & a&= \SI{0.0034}{\kelvin\per\second\squared}\\
  b&=\SI{1.2168}{\kelvin\per\second} & b&=\SI{-0.6725}{\kelvin\per\second} \\
  c&=\SI{294.9701}{\kelvin} & c&=\SI{295.8702}{\kelvin} \\
  \implies T_1(t) &= -0.0116t²+1.2168t+294.9701 &\implies T_2(t) &= 0.0034t² -0.6725t + 295.8702
\end{align*}

\subsection{Bestimmung der Güteziffer}
Zur Bestimmung der Güteziffer errechnen wir die Differantialquotienten $\frac{\symup{d}T_1}{\symup{d}t}$ und $\frac{\symup{d}T_2}{\symup{d}t}$ für vier verschiedene Zeitpunkte:
Die Ableitung ist aus der Ausgleichsrechnung einfach bestimmt.
\begin{equation}
  \frac{\symup{d}T}{\symup{d}t} = 2at + b
\end{equation}
Für $T_1$ ergeben sich die Ableitungen zu:
\begin{gather*}
  \text{bei} t=1.0 : \frac{\symup{d}T_1}{\symup{d}t} = \SI{1.1936}{\kelvin\per\second}\\  %Fehler von Gaus
  \text{bei} t=2.0 : \frac{\symup{d}T_1}{\symup{d}t} = \SI{1.1704}{\kelvin\per\second}\\
  \text{bei} t=3.0 : \frac{\symup{d}T_1}{\symup{d}t} = \SI{1.1471}{\kelvin\per\second}\\
  \text{bei} t=4.0 : \frac{\symup{d}T_1}{\symup{d}t} = \SI{1.1239}{\kelvin\per\second}
\end{gather*}
Für $T_2$ ergeben sich die Ableitungen zu:
\begin{gather*}
  \text{bei} t=1.0 : \frac{\symup{d}T_2}{\symup{d}t} = \SI{-0.6656}{\kelvin\per\second}\\
  \text{bei} t=2.0 : \frac{\symup{d}T_2}{\symup{d}t} = \SI{-0.6588}{\kelvin\per\second}\\
  \text{bei} t=3.0 : \frac{\symup{d}T_2}{\symup{d}t} = \SI{-0.6519}{\kelvin\per\second}\\
  \text{bei} t=4.0 : \frac{\symup{d}T_2}{\symup{d}t} = \SI{-0.6450}{\kelvin\per\second}
\end{gather*}

Mit den Formel \refeq{eqn:gueteziffer} und \refeq{eqn:idguete} lassen sich nun die Güteziffer und die ideale Güteziffer berechnen.
Die Wärmekapazität der Kupferschlangen und des Wasser ergeben sich zu:
\begin{align*}
  m_kc_k &= \SI{750}{\joule\per\kelvin}\\
  mc_w &= 4 \cdot \SI{4184}{\joule\per\kelvin\kilo\gram} = \SI{16736}{\joule\per\kelvin} %muss ich citen ?
\end{align*}

\begin{table}
  \caption{Bestimmung der Güteziffer aus der Messreihe $T_1$}
  \begin{tabular}{
    S[table-format=3.0] %Zeit
    S[table-format=3.4] %guete gerechnet
    S[table-format=3.2] %fehler der gueteziffer 
    S[table-format=3.4] %guete ideal
    S[table-format=2.2] %Abweichung
  }
  \toprule
  {$ t \mathbin{/} \si{\second} $} &
  {$ \nu $} &
  {$ \Delta \nu $} &
  {$ \nu_{\text{id}}$} &
  {$ \text{Abweichung} \mathbin{/} \si{\percent}$} \\
  \midrule
  60&   173.9232&     & 227.8077& 23.65\\
  120&  170.5397&     & 110.1667& 54.80\\
  180&  167.1562&     & 78.5395&  112.83\\
  240&  163.7728&     & 53.4911&  206.17
  \end{tabular}
\end{table}

Der Fehler $\Delta \nu$ wurde mit der Fehlerfortpflanzung von Gauss berechnet.
Man erkennt eine große Abweichung der errechneten Werte für die Güteziffer und den dazu passenden Werten der idealen Güteziffer $\nu_{\text{id}}$.
Dies folgt einerseits aus den Annahmen, es handle sich um einen reversiblen Prozess und der Kompressor arbeite adiabatisch, welche nicht der Realität entsprechen.
Zusätzlich haben wir Energieverluste aus Reibung nicht beachtet. 
In der Diskussion gehen wir nochmal weiter auf mögliche Fehlerquellen ein.

\subsection{Bestimmung des Massendurchsatzes}
Zur Berechnung des Massendurchsatzes benötigen wir die Verdampfungswärme $L$. 
Diese kann wie folgt aus den gemessenen Werte errechnet werden. 
Zuerst erstellt man ein $(p,T)$ Diagramm, wobei die Temperatur als $\frac{1}{T}$ und der Druck als $\log(\frac{p}{p_0})$ mit Umgebungsdruck $p_0$ aufgetragen werden.
Mithilfe einer Ausgleichgerade bestimmt man die Steigung $m$ und kann dann durch $L=-m\cdot R$ die Verdampfungswärme L errechnen. 
$R$ entspricht der allgemeinen Gaskonstante.

\begin{figure}
  \centering
  \includegraphics{TempDruck.pdf}
\end{figure}

Für die Steigung $m$ der Ausgleichsgerade ergibt sich:
\begin{align*}
  \text{für}T_1 \text{und} p_}\text{a}}: m&=\\
  \text{für}T_2 \text{und} p_{\text{b}}: m&=
\end{align*}
Da wir den Massendurchsatz aus den Messwerten bezüglich Reservoir 2 berechnen, werden wir zur Berechnung von $L$ auch die Steigung bezüglich Reservoir 2 benutzen.
Mit Fehlerrechnung nach \cite{fehler}