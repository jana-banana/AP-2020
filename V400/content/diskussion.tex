\section{Diskussion}
\label{sec:Diskussion}

Alle Experimente werden auf einer transparenten Grundplatte durch geführt. Diese wird auf verschiedene ausgedruckte Vorlagen gestellt. Beim Verändern des 
Einfallswinkels ist es aufgrund des leicht schwerfälligen Laufes des Lasers möglich, dass die Platte auf der Vorlage leicht verschoben wird. 
Außerdem werden die Winkel bei der Mesung mit dem Prisma und Beugung am Gitter an einer aufgestellten Pappe abgelesen. Diese wird vorsichtig an die liegende
Winkelskala angepasst, jedoch ist sie nur durch zwei Aufsteller festgehalten und bewegt sich schnell, auch bei leichter Berührung. 
Also sind systematische Fehler definitiv nicht auszuschließen und können durch andere Messapparate minimiert werden. 

\subsection{Reflexionsgesetz}

  In der Auswertung zur Überprüfung des Reflexionsgesetzes wird eine lineare Ausgleichsgerade $\beta = A \cdot \alpha + B$ gelegt. Die Theoriewerte für die 
  Parameter sind $A = 1$ und $B=0$. 
  \begin{align*}
    A_{\text{theo}} &= \num{1} & \implies \increment A &= \SI{-3.54}{\percent} \\
    A_{\text{exp}} &= \num{1.035(5)} 
  \end{align*}
  Es ist zu sehen, dass der Fit-Parameter recht nah an der Theorie liegt. Mögliche Fehlerquellen sind Fehler beim Ablesen. Der Punkt lässt sich auf dem Reflexionsschirm 
  nicht gut ablesen, da dieser mehrere Millimeter breit ist. 

\subsection{Brechungsgesetz} 

  Außerdem wird der Brechungsindex für Plexiglas ermittelt, aus welchem dann auch die Lichtgeschwindigkeit in Plexiglas errechnet wird. 
  \begin{align*}
    n_{\text{Plexiglas, theo}} &= \num{1.49}  \text{\cite{theo_brechzahl}} & \implies \increment n_{\text{Plexiglas}} &= \SI{1.79}{\percent}\\
    n_{\text{Plexiglas, exp}} &= \num{1.463(8)}\\
    \\
    V_{\text{c, Plexiglas, theo}} &= \SI{201202991}{\metre\per\second} \text{\cite{theo_brechzahl}} & \implies \increment v_{\text{c, Plexiglas}} &= \SI{1.83}{\percent}\\
    V_{\text{c, Plexiglas, exp}} &= \SI{204884121(1181566)}{\metre\per\second} 
  \end{align*}
  Der experimentell ermittelte Wert ist sehr nah an dem Theoriewert. Da sich die Lichtgeschwindigkeit in Plexiglas nur aus der Vakuumlichtgeschwindigkeit und 
  dem Brechungsindex herleiten lässt, ist dies hier dann natürlich auch der Fall. 

\subsection{Planparallele Platten}

  Bei der verschiedenen Berechnung des Strahlenversatzes fallen keine großen Unterschiede zwischen den errechneten Werten auf. Der gemessene Brechungswinkel ist dem
  errechneten Winkel sehr nah, welches nochmal zeigt, dass der Wert für den Brechungsindex recht gut ermittelt wird. Es ist zu sagen, dass die Methode über den 
  gemessenen Winkel natürlich mehr anfällig für Messfehler ist, jedoch wird der Brechungswinkel mit dem experimentell ermittelten Brechungsindex berechnnet, 
  sodass sich hier die Fehler fortpflanzen.

\subsection{Prisma}

  Hier werden die Ablenkungen $\delta$ für die verschiedenen Einfallswinkel berechnet. Aus den gemessenen Einfalls- und Ausfallswinkel sind die Brechungswinkel 
  zu berechnen, deren Summe eigentlich konstant dem brechenden Winkel $\gamma = \SI{60}{\degree}$ entspricht.Hier ist das jedoch nicht der Fall. Dies kann 
  ein Resultat von ungenau abgelesenen Winkeln sein, ein geringer Fehler kann auch von den numerischen Näherungen der Berechnungen vom Rechner herrühren. 


\subsection{Beugung am Gitter}

  Die Ermittlung der Wellenlängen der Laser durch Beugung am Gitter hat diese Ergebnisse ergeben:
  \begin{align*}
    \lambda_{\text{grün, theo}} &= \SI{532}{\nano\metre} \text{\cite{anleitung}} & \implies \increment \lambda_{\text{grün}} &= \SI{-0.27}{\percent} \\
    \lambda_{\text{grün, exp}} &= \SI{533(8)}{\nano\metre}\\
    \\
    \lambda_{\text{rot, theo}} &= \SI{635}{\nano\metre} \text{\cite{anleitung}} & \implies \increment \lambda_{\text{rot}} &= \SI{-0.83}{\percent} \\
    \lambda_{\text{rot, exp}} &= \SI{640(5)}{\nano\metre}
  \end{align*}
  Dies sind sehr gute Ergebnisse, besonders bei Erinnerung an den Versuchsaufbau. 

% \begin{table}
%     \centering
%     \caption{Ein Vergleich der experimentell ermittelten und theoretischen Werten. }
%     \label{tab:vergleich}
%     \begin{tabular}{c c@{${}\pm{}$} c c c }
%       \toprule
%       {Wert} & {experimenteller Wert} & {theoretischer Wert} & {Abweichung $/ \si{\degree} $} \\
%       \midrule
%       $\text{Parameter} A$ & 1.035 & 0.005 & 1 & -3.54 \\
%       $n_{\text{Plexiglas}}$ & 1.463 & 0.008 & 1.49  \cite{theo_brechzahl} & 1.79 \\
%       $v_{\text{c, Plexiglas}}$ & 204884121 & 1181566$\si{\metre\per\second}$ & 201202991 $\si{\metre\per\second}$& -1.83 \\
%       $\lambda_{\text{grün}}$ & 533 & 8 $\si{\nano\metre}$ & 532 $\si{\nano\metre}$ \cite{anleitung} & -0.27 \\
%       $\lambda_{\text{rot}}$ & 640 & 5 $\si{\nano\metre}$ & 635 $\si{\nano\metre}$ \cite{anleitung} & -0.83 \\
%       \bottomrule 
%     \end{tabular}
%   \end{table}