\section{Durchführung}
\label{sec:Durchführung}
Alle im folgenden beschriebenen Schaltungen werden mit einer Sinus-Spannug der Frequenz $\SI{1000}{\hertz}$ gespeist.
Zum Ablesen der Brückenspannung ist ein Oszilloskop angebracht.
\subsection{Wheatstonesche Brücke}
Aus den uns vorliegenden Bauteilen wird die Wheatstonesche Brücke aufgebaut.
Da uns nur der Quotient $\frac{R_3}{R_4}$ interessiert werden die Widerstände als Potentiometer mit einem Gesamtwiderstand von $\SI{1}{\kilo\ohm}$.
Ein unbekannter Widerstand wird in die Schaltung in der Position von $ R_{\text{x}}$ eingebaut.
Nun wird das Potentiometer so eingestellt, dass die Brückenspannung minimal wird, die Werte werden notiert.
Zur möglichst genauen Bestimmung wechseln wir den Widerstand $R_2$ 2 mal aus und wiederholen den Vorgang.
Nach dem gleichen Prinzip wird noch ein anderer Widerstand ausgemessen.

\subsection{Kapazitätsmessbrücke}
Es wird eine Kapazitätsmessbrücke aufgebaut.
Ein Potentiometer wird für die Widerstände $R_3$ und $R_4$ eingebaut, der $R_2$ wird durch einen verstellbaren Widerstand realisiert.
Das Referenzbauteil ist in diesem Fall der Kondensator $C_2$. \\
Zu Bestimmen ist eine Kapazität $C_{\text{x}}$ mit dem Innenwiderstand $R_{\text{x}}$.
Zu Beginn sollten die verstellbaren Widerstände auf null stehen, dann erhöht man die Widerstände langsam abwechselnd, bis die Brückenspannung minimal ist.
Die Werte werden notiert und für eine andere Kapazität $C_2$ wird der Messvorgang wiederholt.

\subsection{Induktivitätsmessbrücke}
Die Induktivitätsmessbrücke wird nach ... aufgebaut.
Das Potientiometer und der verstellbare Widerstand $R_2$ bleiben, das Referenzbauteil wird eine Induktivität $L_2$. 
Mit dieser Brücke wollen wir eine Induktivität $L_{\text{x}}$ mit dem Innenwiderstand $R_{\text{x}}$ ausmessen.
Dafür werden wieder die verstellbaren Widerstände abwechselnd erhöht, bis die Brückenspannung minimal wird.
Da wir keine andere Induktivität haben, können wir $L_2$ nicht austauschen und somit die unbekannte Induktivität nur durch eine Messung bestimmen.
Anschließend wird der Messvorgang für eine weitere Induktivität wiederholt.

\subsection{Wien-Robinson-Brücke}
Es wird die Schaltung aufgebaut.
Die Frequenz wird nun im Bereich von 20-$\SI{30000}{\hertz}$ variiert.
Es werden die Speisespannung $U_{\text{S}}$ und die Brückenspannung $U_{\text{Br}}$ bei verschiedenen Frequenzen gemessen und notiert.