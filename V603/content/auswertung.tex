\section{Auswertung}
\label{sec:Auswertung}

\subsection{Aufnahme des Emissionsspektrum der Kupfer-Röntgenröhre}

  Es werden in einem Winkelbereich von $ \SI{8}{\degree} \leq \theta \leq \SI{25}{\degree} $ die Zählraten bei einer Integrationszeit von $t=\SI{10}{\second}$ 
  aufgenommen, die Daten sind in der \autoref{tab:emissionsspektrum} zu finden. 

  \begin{table}
  \centering
  \caption{Die Messwerte des Emissionsspektrum der Kupfer-Röntgenröhre.}
  \label{tab:emissionsspektrum}
  \begin{tabular}{S[table-format=2.1] S[table-format=3.1] S[table-format=2.1] S[table-format=3.1] S[table-format=2.1] S[table-format=3.1] S[table-format=2.1] S[table-format=4.1] S[table-format=2.1] S[table-format=4.1]}
    \toprule
    $ \theta \, \mathbin{/} \si{\degree}$ & $ N \, \mathbin{/} \si{Imp\per\second}$ & 
    $ \theta \, \mathbin{/} \si{\degree}$ & $ N \, \mathbin{/} \si{Imp\per\second}$ &
    $ \theta \, \mathbin{/} \si{\degree}$ & $ N \, \mathbin{/} \si{Imp\per\second}$ &
    $ \theta \, \mathbin{/} \si{\degree}$ & $ N \, \mathbin{/} \si{Imp\per\second}$ &
    $ \theta \, \mathbin{/} \si{\degree}$ & $ N \, \mathbin{/} \si{Imp\per\second}$ \\
    \cmidrule(lr){1-2}\cmidrule(lr){3-4}\cmidrule(lr){5-6}\cmidrule(lr){7-8} \cmidrule(lr){9-10}
    8.0 	&	323.0 &    11.5	&	406.0 &    14.9	&	248.0 &     18.3	&	166.0  &      21.7	&	164.0 \\   
    8.1 	&	316.0 &    11.6	&	404.0 &    15.0	&	253.0 &     18.4	&	173.0  &      21.8	&	180.0 \\   
    8.2 	&	326.0 &    11.7	&	405.0 &    15.1	&	257.0 &     18.5	&	167.0  &      21.9	&	179.0 \\   
    8.3 	&	340.0 &    11.8	&	400.0 &    15.2	&	248.0 &     18.6	&	169.0  &      22.0	&	191.0 \\   
    8.4 	&	335.0 &    11.9	&	383.0 &    15.3	&	242.0 &     18.7	&	160.0  &      22.1	&	232.0 \\   
    8.5 	&	343.0 &    12.0	&	389.0 &    15.4	&	249.0 &     18.8	&	159.0  &      22.2	&	300.0 \\   
    8.6 	&	350.0 &    12.1	&	382.0 &    15.5	&	246.0 &     18.9	&	157.0  &      22.3	&	536.0 \\   
    8.7 	&	350.0 &    12.2	&	372.0 &    15.6	&	252.0 &     19.0	&	149.0  &      22.4	&	4128.0\\    
    8.8 	&	366.0 &    12.3	&	376.0 &    15.7	&	236.0 &     19.1	&	153.0  &      22.5	&	5050.0\\    
    8.9 	&	357.0 &    12.4	&	385.0 &    15.8	&	234.0 &     19.2	&	150.0  &      22.6	&	4750.0\\    
    9.0 	&	371.0 &    12.5	&	384.0 &    15.9	&	231.0 &     19.3	&	147.0  &      22.7	&	4571.0\\    
    9.1 	&	371.0 &    12.6	&	382.0 &    16.0	&	215.0 &     19.4	&	150.0  &      22.8	&	4097.0\\    
    9.2 	&	372.0 &    12.7	&	373.0 &    16.1	&	217.0 &     19.5	&	148.0  &      22.9	&	901.0 \\   
    9.3 	&	364.0 &    12.8	&	376.0 &    16.2	&	227.0 &     19.6	&	149.0  &      23.0	&	244.0 \\   
    9.4 	&	381.0 &    12.9	&	373.0 &    16.3	&	214.0 &     19.7	&	143.0  &      23.1	&	179.0 \\   
    9.5 	&	379.0 &    13.0	&	375.0 &    16.4	&	217.0 &     19.8	&	153.0  &      23.2	&	151.0 \\   
    9.6 	&	393.0 &    13.1	&	366.0 &    16.5	&	210.0 &     19.9	&	182.0  &      23.3	&	145.0 \\   
    9.7 	&	375.0 &    13.2	&	354.0 &    16.6	&	211.0 &     20.0	&	291.0  &      23.4	&	130.0 \\   
    9.8 	&	391.0 &    13.3	&	341.0 &    16.7	&	206.0 &     20.1	&	1127.0 &      23.5	&	121.0 \\    
    9.9 	&	395.0 &    13.4	&	326.0 &    16.8	&	205.0 &     20.2	&	1599.0 &      23.6	&	126.0 \\    
    10.0	&	402.0 &    13.5	&	318.0 &    16.9	&	198.0 &     20.3	&	1533.0 &      23.7	&	117.0 \\    
    10.1	&	405.0 &    13.6	&	305.0 &    17.0	&	203.0 &     20.4	&	1430.0 &      23.8	&	112.0 \\    
    10.2	&	390.0 &    13.7	&	296.0 &    17.1	&	199.0 &     20.5	&	1267.0 &      23.9	&	110.0 \\    
    10.3	&	398.0 &    13.8	&	286.0 &    17.2	&	198.0 &     20.6	&	425.0  &      24.0	&	105.0 \\   
    10.4	&	400.0 &    13.9	&	285.0 &    17.3	&	191.0 &     20.7	&	241.0  &      24.1	&	106.0 \\   
    10.5	&	418.0 &    14.0	&	274.0 &    17.4	&	192.0 &     20.8	&	225.0  &      24.2	&	107.0 \\   
    10.6	&	401.0 &    14.1	&	264.0 &    17.5	&	184.0 &     20.9	&	192.0  &      24.3	&	95.0  \\  
    10.7	&	410.0 &    14.2	&	266.0 &    17.6	&	191.0 &     21.0	&	188.0  &      24.4	&	94.0  \\  
    10.8	&	408.0 &    14.3	&	270.0 &    17.7	&	188.0 &     21.1	&	172.0  &      24.5	&	100.0 \\   
    10.9	&	409.0 &    14.4	&	255.0 &    17.8	&	181.0 &     21.2	&	168.0  &      24.6	&	91.0  \\  
    11.0	&	414.0 &    14.5	&	255.0 &    17.9	&	185.0 &     21.3	&	169.0  &      24.7	&	85.0  \\  
    11.1	&	420.0 &    14.6	&	260.0 &    18.0	&	184.0 &     21.4	&	166.0  &      24.8	&	88.0  \\  
    11.2	&	417.0 &    14.7	&	251.0 &    18.1	&	179.0 &     21.5	&	170.0  &      24.9	&	83.0  \\  
    11.3	&	417.0 &    14.8	&	250.0 &    18.2	& 180.0 &     21.6	&	174.0  &      25.0	&	85.0  \\  
    11.4	&	409.0 \\
    \bottomrule
  \end{tabular}
  \end{table}

  \noindent Die Daten aus \autoref{tab:emissionsspektrum} sind in der \autoref{fig:emissionsspektrum} eingezeichnet. Es ist das Bremsspektrum der Röntgenstrahlung 
  zu erkennen, sowie die charakteristischen $K_{\alpha}$- und $K_{\beta}$-Peaks. Diese markanten Punkte sind in der Abbildung gekennzeichnet. Diese Stellen lauten:
  \begin{align*}
    \text{für $K_{\alpha}$:} \quad \theta_{\alpha} &= \SI{22.5(1)}{\degree} & \implies E_{\alpha} &= \SI{8043(34)}{\electronvolt}\\
                                    N_{\alpha} &= \SI{5050}{Imp\per\second} \\
                                    \\
    \text{für $K_{\beta}$:} \quad \theta_{\beta} &= \SI{20.2(1)}{\degree} & \implies E_{\beta} &= \SI{8910(40)}{\electronvolt}\\
                                    N_{\beta} &= \SI{1599}{Imp\per\second} 
  \end{align*}
  Die Energien werden mithilfe der Bragg-Bedingung und dem Zusammenhang für Photonen $E = \frac{\symup{h} \symup{c}}{\lambda}$ über die Formel berechnet
  \begin{equation*}
    E = \frac{\symup{h} \cdot \symup{c}}{2 d \cdot \sin(\theta)}. 
  \end{equation*}
  Hier beschreibt $h$ das Planck'sche Wirkungsquantum und $c$ die Lichtgeschwindigkeit. 
  

  \begin{figure}[H]
    \centering
    \includegraphics[width=\textwidth]{build/emissionsspektrum.pdf}
    \caption{Die Messdaten des Emissionsspektrum aufgetragen gegen den Winkel.}
    \label{fig:emissionsspektrum}
  \end{figure}

\subsection{Bestimmung der Transmission als Funktion der Wellenlänge}
\label{subsec:T(lambda)}

  Zur Bestimmung der Transmission wird die Zählrate für Winkel im Bereich von $\SI{7}{\degree}-\SI{10}{\degree}$ bei einer Integrationszeit von 
  $t = \SI{200}{\second}$ einmal mit Aluminiumabsorber ($N_{\text{Al}}$) und einmal ohne ($N_0$) aufgenommen, in der \autoref{tab:aluabsorber} sind die Messdaten 
  zu finden.

  \begin{table}[H]
    \centering
    \caption{Die Daten der Zählratenmessung bei verändertem Winkel mit und Aluminiumabsorber.}
    \label{tab:aluabsorber}
    \begin{tabular}{S[table-format=1.1] S[table-format=3.1] S[table-format=3.1] S[table-format=2.1] S[table-format=3.1] S[table-format=3.1]} %{c c c c c c }
      \toprule
      {$ \theta \, \mathbin{/} \si{\degree}$} & {$ N_0 \, \mathbin{/} \si{Imp\per\second}$} &  {$ N_{\text{Al}} \, \mathbin{/} \si{Imp\per\second}$} & 
      {$ \theta \, \mathbin{/} \si{\degree}$} & {$ N_0 \, \mathbin{/} \si{Imp\per\second}$} &  {$ N_{\text{Al}} \, \mathbin{/} \si{Imp\per\second}$} \\
      % $\theta \mathbin{/} \si{\degree}$ & $N_0 \, \mathbin{/} \si{Imp\per\second} $ & $ N_{\text{Al}} \,  \mathbin{/} \si{Imp\per\second}  $ & %@{${}\pm{}$}
      % $\theta \mathbin{/} \si{\degree}$ & $N_0 \, \mathbin{/} \si{Imp\per\second} $ & $ N_{\text{Al}} \,  \mathbin{/} \si{Imp\per\second}  $ \\ %\multicolumn{2}{c}{}
      \cmidrule(lr){1-3} \cmidrule(lr){4-6}
      7.0 &   226.0 &   113.5 &     8.6   &   328.5 &   100.0   \\
      7.1 &   232.0 &   112.0 &     8.7   &   332.5 &   100.5   \\
      7.2 &   240.5 &   112.0 &     8.8   &   337.0 &   97.5    \\
      7.3 &   248.0 &   113.5 &     8.9   &   340.5 &   95.0    \\
      7.4 &   255.0 &   115.0 &     9.0   &   348.0 &   92.5    \\
      7.5 &   262.0 &   113.5 &     9.1   &   350.0 &   89.5    \\
      7.6 &   269.0 &   113.0 &     9.2   &   353.0 &   88.0    \\
      7.7 &   276.0 &   114.5 &     9.3   &   356.5 &   84.5    \\
      7.8 &   281.0 &   114.0 &     9.4   &   359.0 &   83.0    \\
      7.9 &   289.5 &   112.0 &     9.5   &   363.5 &   81.0    \\
      8.0 &   295.0 &   109.5 &     9.6   &   367.0 &   78.5    \\
      8.1 &   300.0 &   109.0 &     9.7   &   369.0 &   76.0    \\
      8.2 &   308.5 &   108.0 &     9.8   &   370.5 &   74.0    \\
      8.3 &   311.0 &   106.0 &     9.9   &   375.0 &   72.0    \\
      8.4 &   317.0 &   104.5 &     10.0  &   375.5 &   68.5    \\
      8.5 &   324.0 &   101.5 \\ 
      \bottomrule        
    \end{tabular}
  \end{table}

  \noindent In der \autoref{tab:aluabsorber} sind auch die Messunsicherheiten eingetragen, es liegt eine Poisson-Verteilung bei der Anzahl der Röntgenquanten 
  vor. Die Zählraten werden nun mit einer Totzeitkorrektur bereinigt
  \begin{equation*}
    I = \frac{N}{1 - \tau \cdot N} \, , 
  \end{equation*}
  wobei $\tau = \SI{90}{\micro\second}$ die Totzeit des benutzten Geiger-Müller-Zählrohr ist. Die Transmission berechnet sich dann über $T = \frac{I_{\text{Al}}}{I_0}$.

  \noindent Die Wellenlänge $\lambda $ berechnet sich aus dem Winkel über die Bragg Bedingung , die Beugungsordnung ist $n=1$ und die Gitterkonstante für einen 
  LiF-Kristall $d = \SI{201.4}{\pico\metre}$. Die Transmission ist in der \autoref{fig:transmission} gegen die Wellenlänge aufgetragen.

  \begin{figure}[H]
    \centering
    \includegraphics[width = \textwidth]{build/transmission.pdf}
    \caption{Die Transmissions aufgetragen gegen die Wellenlänge $\lambda$ inklusive einer linearen Ausgleichsrechnung.}
    \label{fig:transmission}
  \end{figure}

  \noindent Es wird außerdem eine lineare Ausgleichsrechnung $T = a \cdot \lambda + b$ mit python durchgeführt, welche auch in \autoref{fig:transmission} eingezeichnet 
  sind. Die Koeffizienten ergeben sich zu:
  \begin{align*}
    a &= \SI{-15.195(239)e9}{1\per\metre}\\
    b &= \num{1.225(14)}
  \end{align*}


\subsection{Bestimmung der Compton-Wellenlänge} %!!!!! WILL ICH EINE TABELLE ODER NICHT ? nein, tabelle mag mich nicht

  Die Impulse werden bei einer Integrationszeit von $ t = \SI{300}{\second}$ für $I_0$ ohne Aluminiumabsorber, für $I_1$ mit Aluminiumabsorber zwischen der 
  Röntgenröhre und dem Plexiglas-Streuer und für $I_2$ mit Aluminiumabsorber zwischen dem Plexiglas-Streuer und dem Geiger-Müller-Zählrohr aufgenommen. 
  %------------Variante 1----------------
  Es ergeben sich 
  folgende Werte:
  \begin{align*}
    I_0 &= \num{2731(50)}\\
    I_1 &= \num{1180(34)}\\
    I_2 &= \num{1024(32)}
  \end{align*}

  \noindent Aus diesen Messwerten lassen sich die Transmission berechnen:
  \begin{align*}
    T_1 &= \frac{I_1}{I_0} = \num{0.432(15)}\\
    T_2 &= \frac{I_2}{I_0} = \num{0.375(14)}
  \end{align*}

  \noindent Nun wird die Wellenlänge $\lambda$ zu den Transmissionen errechnet. Die lineare Abhängigkeit $T(\lambda) = a \cdot \lambda + b $, 
  welche in der Teilaufgabe \ref{subsec:T(lambda)} ermittelt wird, wird so genutzt
  \begin{equation*}
    \lambda = \frac{(T - b)}{a}
  \end{equation*}
  mit den Koeffizienten
  \begin{align*}
    a &= \SI{-15.195(239)e9}{1\per\metre}\\
    b &= \num{1.225(14)}.
  \end{align*}
  Es ergeben sich die Werte:
  \begin{align*}
    \lambda_1 = \SI{52.2(16)}{\pico\metre}\\
    \lambda_2 = \SI{55.9(16)}{\pico\metre}
  \end{align*}

  \noindent Damit lässt sich die Compton-Wellenlänge bestimmen: $\lambda_{\text{c}} = \lambda_2 - \lambda_1 = \SI{3.8(11)}{\pico\metre}$. 

  %------------Variante 2 ---------> nope, taking the other one
  % \begin{table}
  %   \centering
  %   \caption{Die Tabelle mit den gemessenen Impulsen und den damit berechneten Werte.}
  %   \label{tab:c)}
  %   \begin{tabular}{S[table-format=1.0] S[table-format=1.3] S[table-format=2.1] S[table-format=2.1] @{${}\pm{}$} S[table-format=1.1]}
  %     \toprule
  %     {Nr} & {$ I \, / \si{Imp}$} & {$ T = \frac{I}{I_0}$} & \multicolumn{2}{c}{$ \lambda \, / \si{\pico\metre}$} \\
  %     \midrule
  %     0 & 2731 \\
  %     1 & 1180  &  0.432   & 52.2&1.2  \\
  %     2 & 1024  &  0.375   & 52.9&1.3  \\
  %     \bottomrule
  %   \end{tabular}
  % \end{table}

  % \noindent In der \autoref{tab:c)} sind Impulse aufgeschrieben. Die Transmissionen werden über die Formel
  % \begin{equation*}
  %   T_i = \frac{I_i}{I_0}
  % \end{equation*}
  % berechnet. Mithilfe des linearen Zusammenhangs $T(\lambda) = a \cdot \lambda + b $ aus dem Aufgabenteil \ref{subsec:T(lambda)} wird die Wellenlänge zu 
  % \begin{align*}
  %   \lambda &= \frac{(T - b)}{a}\\
  %   \\
  %   \text{mit:} \quad     a &= \SI{15.19(24)e9}{1\per\metre}\\
  %                         b &= \num{1.225(14)}
  % \end{align*} 
  % Die Compton-Wellenlänge ergibt sich nun aus der Differenz der ermittelten Wellenlängen: $\lambda_{\text{c}} = \lambda_2 - \lambda_1 = \SI{3.76(6)}{\pico\metre}$. 