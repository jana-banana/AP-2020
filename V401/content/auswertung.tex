\section{Auswertung}
\label{sec:Auswertung}

\subsection{Bestimmung der Wellenlänge des Lasers}
\label{subsec:Wellenlänge}

Mithilfe von der Gleichung \eqref{eqn:forlambda} wird die Wellenlänge des Lasers bestimmt.
Da die Verschiebung des Spiegels $\increment d$ nicht genau der Mikrometerschraube entspricht, wird die Formel \eqref{eqn:forlambda} um $\frac{1}{Ü}$ erweitert.
Hierbei ist $Ü$ die Hebelübersetzung und beträgt $5.046$.
Die aufgenommenen und verwendeten Messwerte sowie die Ergebnisse sind in \autoref{tab:Messwerte und Wellenlänge} zu finden.

\begin{table}
  \centering
  \caption{Die aufgenommenen Messwerte und die daraus bestimmte Wellenlänge.}
  \label{tab:Messwerte und Wellenlänge}
  \begin{tabular}{c c c}
    \toprule
    {$\increment d$ [$\si{\milli\metre}$]} & {$z$} & {$\lambda$ [$\si{\milli\metre}$]} \\
    \midrule
    5.0     &       3016 $\pm$ 4    &   657.0848 $\pm$ 0.8715\\
    5.0     &       2972 $\pm$ 4    &   666.8128 $\pm$ 0.8975\\
    5.0     &       2976 $\pm$ 4    &   665.9166 $\pm$ 0.8950\\
    5.0     &       2963 $\pm$ 4    &   668.8383 $\pm$ 0.9029\\
    5.1     &       2988 $\pm$ 4    &   676.5071 $\pm$ 0.9056\\
    4.9     &       2964 $\pm$ 4    &   655.2403 $\pm$ 0.8843\\
    5.1     &       2987 $\pm$ 4    &   676.7335 $\pm$ 0.9062\\
    5.1     &       2974 $\pm$ 4    &   679.6917 $\pm$ 0.9142\\
    5.1     &       2967 $\pm$ 4    &   681.2953 $\pm$ 0.9185\\
    4.9     &       2967 $\pm$ 4    &   654.5778 $\pm$ 0.8825\\
    \bottomrule
  \end{tabular}
\end{table}

\noindent
Durch eine Mittlung ergibt sich für die Wellenlänge
\begin{equation*}
  \lambda = \SI{668.27 \pm 0.28}{\nano\metre}.
\end{equation*}

\subsection{Bestimmung des Brechungsindexes von Luft}
\label{Brechungsindex}

Nach der Formel \eqref{eqn:forindex} kann der Brechungsindex von Luft bestimmt werden.
Hierzu werden folgende Werte benötigt:

\begin{align*}
  \text{Größe der Messzelle :}&   &b &= \SI{50e-3}{\metre} \\
  \text{Umgebungstemperatur :}&   &T &= \SI{293.15}{\kelvin} \\
  \\
  \text{Normalbedingungen} & \\
  \text{Normaldruck :}&           &p_0 &= \SI{1.0132}{\bar} \\
  \text{Normaltemperatur :}&      &T_0 &= \SI{273.15}{\kelvin}
\end{align*}
\noindent
Die Messwerte sowie den dazu berechneten Brechungsindex sind in \autoref{tab:Messwerte und Brechungsindex} zu finden.

\begin{table}
  \centering
  \caption{Messwerte und die daraus bestimmten Brechungsindex.}
  \label{tab:Messwerte und Brechungsindex}
  \begin{tabular}{c c c}
    \toprule
    {$\increment p$ [$\si{\bar}$]} & {$z$} & {$n$} \\
    \midrule
    600     &   22  &  1.1999 $\pm$ 0.03634\\
    600     &   33  &  1.2998 $\pm$ 0.03634\\
    600     &   52  &  1.4724 $\pm$ 0.03634\\
    600     &   45  &  1.4088 $\pm$ 0.03634\\
    520     &   54  &  1.5660 $\pm$ 0.04193\\
    600     &   43  &  1.3906 $\pm$ 0.03634\\
    600     &   56  &  1.5087 $\pm$ 0.03634\\
    600     &   34  &  1.3089 $\pm$ 0.03634\\
    600     &   44  &  1.3997 $\pm$ 0.03634\\
    \bottomrule
  \end{tabular}
\end{table}
\noindent
Durch eine Mittlung ergibt sich für den Brechungsindex von Luft:
\begin{equation*}
  n = 1.395 \pm 0.012.
\end{equation*}