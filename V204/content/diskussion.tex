\section{Diskussion}
\label{sec:Diskussion}
In der statischen Methode wurden die Temperaturen an den vom Peltier-Element entfernten Punkten nach $\SI{700}{\second}$ genauer betrachtet.
\begin{table}
    \centering
    \caption{Die Temperatur an den vom Peltier-Element entfernten Messstellen nach $\SI{700}{\second}$.}
    \label{tab:diskussion_stat}
    \begin{tabular}{S[table-format= 15]
                    S[table-format=2.2]
                    S[table-format=2.2]
                    S[table-format=2.2]
                    }
    \toprule
    {Metall}&{$T_{\text{fern}, \SI{700}{\second}} [\si{\celsius}]$}&{$T_{\text{fern}, \SI{0}{\second}} [\si{\celsius}]$}& {$\Delta T_{\text{fern}} [\si{\celsius}]$}\\
    \midrule
    \text{Messing (breit)} & 45.76 & 16.67 & 29.09 \\
    \text{Messing (schmal)}& 43.16 & 16.98 & 26.18 \\
    \text{Aluminium} & 48.46 & 15.47 & 32.99 \\
    \text{Edelstahl} & 34.21 & 19.23 & 14.98 \\
    \bottomrule
    \end{tabular}
\end{table}
In der Tabelle \ref{tab:diskussion_stat} ist an den Temperaturdifferrenz $\Delta T$ gut zu erkennen, dass von den betrachteten Metallen Aluminium die beste Wärmeleitfähigkeit hat.
Da der Wärmestrom von der Querschnittsfläche abhängig ist, ist es verständlich, dass der breite Messingstab eine größere Temperaturerhöhung hat als der schmale Stab.\\
\\
In der Tabelle \ref{tab:Wärmestrom} und der Abbildung \ref{fig:plot_diff} ist zu erkennen, dass der Wärmestrom $\frac{\symup{d}Q}{\symup{d}t}$ mit der Zeit abgenommen hat und sich einem konstanten Wert annähert.
Der schnelle Anstieg der Temperaturdifferenz und damit des Wärmestromes am Anfang ist damit zu erklären, dass das eine Ende des Stabes von Peltier-Element erhitzt wurde und die Wärme noch nicht am anderen Ende angekommen ist.
Das Fallen der Temperaturdifferenz kommt dann daher, dass die an den Stab gegebene Wärme an der entfernten Messstelle angekommen ist.
Da das Peltier-Element den Stab konstant erhitzt, ist der konstante Grenzwert zu erklären.\\
\\
\\
Aus der dynamischen Methode (Angström-Methode) wurde für Messing, Aluminium und Edelstahl die Wärmeleitfähigkeit $\kappa $ errechnet.
\begin{table}[H]
    \centering
    \caption{Vergleich der im Experiment ermittelten Größen zu ihren Literaturwerten.}
    \label{tab:diskussion}
    \begin{tabular}{S[table-format=15]
                    S[table-format=3.3]
                    S[table-format=9]
                    S[table-format=1.3]}
    \toprule
    {Metall}&{$\kappa_{\text{Messung}} [\si{\watt\per\metre\per\kelvin}]$}&{$\kappa_{\text{Literatur}} [\si{\watt\per\metre\per\kelvin}]$ \cite{chemie.de}}&{Abweichung $[\si{\percent}]$}\\
    \midrule
    \text{Messing (breit)} & 117.264 & 120 & 2.28 \\
    \text{Aluminium}& 231.646 & 237 & 2.259 \\
    \text{Edelstahl}& 16.145 & 15 bis 21\\
    \bottomrule 
    \end{tabular}
\end{table}
In der Tabelle \ref{tab:diskussion} sind die aus der Messung ermittelten Werte für $\kappa $ neben den Literaturwerten aus der Tabelle \ref{tab:} aufgelistet. 
An den prozentualen Abweichungen bei Messing und Aluminium ist zu erkennen, dass die Messung gut funktioniert hat. 
Auch der im Experiment ermittelte Wert für die Wärmeleitfähigkeit von Edelstahl liegt in dem angegebenen Bereich der Literatur.
Die Abweichungen sind durch die ungenaue Ermittlung der Amplituden und der Phasendifferenz zu erklären.
Außerdem ist anzumerken, dass das Fenster im Raum die meiste Zeit offen war, welches trotz der Isolierung Auswirkungen haben kann.
Die Temperatur im Raum war daher eher kühler als Zimmertemperatur, aber in de Zeit relativ konstant. \\
In der \autoref{fig:messing_dyn} und der \autoref{fig:aluminium_dyn} ist in dem Temperaturverlauf der nah am Peltier-Element liegenden Messstelle in der vierten Amplitude jeweils eine kleine Abweichung kurz nach dem Hochpunkt zu erkennen.
Es ist anzunehmen, dass es sich hier um den Moment in der Messung handelt, wo aus Unsicherheit, in welche Stellung der Regler gebracht werden muss, dieser mehrfach bewegt wurde. \\
\\
