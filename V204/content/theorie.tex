\section{Zielsetzung}
\label{Zielsetzung}
In diesem Versuch soll die Wärmeleitfähigkeit $\kappa $ von Aluminium, Messing und Edelstahl bestimmt werden.
Dabei wird zuerst die statische Methode angewendet und danach die dynamische Methode, das Angström-Meßverfahren.

\section{Theorie}
\label{Theorie}
Ein Körper im Temperaturungleichgewicht gleicht diesen Unterschied durch Konvektion, Wärmestrahlung oder Wärmeleitung aus.
Im folgenden wird die Wärmeleitung untersucht.
Bei einem Temperaturungleichgewicht eines Stabes ist ein Ende wärmer als das andere.
Dann fließt in der Zeit $dt$ die Wärmemenge $dQ$:
\begin{equation}\label{eqn:Wärmemenge}
    dQ = -\kappa A \frac{\partial T}{\partial x}dt .
\end{equation}
Wobei $A$ für den Querschnitt des Stabes und $\kappa$ für die materialabhängige Wärmeleitfähigkeit steht.
Die Gleichung \eqref{eqn:Wärmemenge} hängt somit nur von den Eigenschaften des Stabes ab.

Die eindimensionale Wärmeleitungsgleichung kann mithilfe der Kontinuitätsgleichung aufgestellt werden:
\begin{equation*}
    \frac{\partial T}{\partial t} = \frac{\kappa}{\rho c} \frac{\partial^2 T}{\partial x^2}
\end{equation*}
mit der Dichte $\rho$ und der spezifischen Wärme $c$ des Stabes.
$\frac{\kappa}{\rho c}$ wird auch Temperaturleitfähigkeit $\sigma_\text{T}$ genannt.

Wird der Stab nun periodisch erwärmt und abgekühlt, folgt eine räumliche und zeitliche Temperaturwelle:
\begin{equation*}\label{eqn:Tempwelle}
    T(x,t) = T_\text{max} \exp(-\sqrt{\frac{\omega \rho c}{2\kappa}}x)\cos(\omega t - \sqrt{\frac{\omega \rho c}{2\kappa}}x)
\end{equation*}
Daraus folgt die Phasengeschwindigkeit $v$:
\begin{equation*}
    v = \frac{\omega}{k} 
    = \frac{\omega}{\sqrt{\frac{\omega \rho c}{2\kappa}}}
\end{equation*}
mit der Kreisfrequenz $\omega$ und der Phase $\Phi$
\begin{align*}
    \omega     &= \frac{2\pi}{T^*}   &  \Phi      &= \frac{2\pi \Delta t}{T^*} \\
    T^* &\widehat{=} \: \text{Periodendauer}\\
\end{align*}
gilt für die Wärmeleitfähigkeit $\kappa$:
\begin{equation}
    \label{eqn:kappa}
    \kappa = \frac{\rho c (\Delta x)^2}{2\Delta t \ln (\frac{A_{\text{nah}}}{A_{\text{fern}}})} .
\end{equation}
$A_{\text{nah}}$ und $A_{\text{fern}}$ sind jeweils die Amplituden an den Messstellen $x_{\text{nah}}$ und $x_{\text{fern}}$.
Wird nun die Gleichung \ref{eqn:kappa} und die Formel für die Phasengeschwindigkeit $v = \frac{\lambda}{T^*}$ verwendet, 
folgt für die Wellenlänge $\lambda$:
\begin{equation}
    \label{eqn:lambda}
    \lambda = \frac{2\pi}{\sqrt{\frac{2 \pi \Delta t \ln(\frac{A_{\text{nah}}}{A_{\text{fern}}})}{T^* (\Delta x)^2}}}
\end{equation}
$\Delta x$ ist hierbei der Abstand der Messstellen $x_{\text{nah}}$ und $x_{\text{fern}}$, $\Delta t$ die Phasendifferenz der Temperaturwelle an diesen Messstellen.
