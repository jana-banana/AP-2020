\section{Diskussion}
\label{sec:Diskussion}
Sinusgenerator nicht gut einstellbar: Bei drehung keine garantie von änderung in diese Richtung
Andere einstellungen als in der Anleitung
keine feine einstellbarkeit des Sinusgenerators
schwierigkeiten das ergebnis am AC-Millivoltmeter abzulesen
teilweise unterschiedliche Ausgangsspannungen bei denselben Frequenzen

Die Untersuchung des Selektivverstärkers wurde durch den Sinusgenerator erschwert.
Bei jedem Versuch die Frequenz zu ändern, sprang der Sinusgenerator durch den ganzen Frequenzbereich.
Das Ablesen am AC-Millivoltmeter stellte sich genau so schwierig dar, da der Zeiger sich oft durch die ganze Skala frei bewegte.
Dadurch konnten weder klein- noch großschrittige Untersuchungen vorgenommen werden.
Es wurden Messpaare aufgenommen, die über einen kleinen Zeitraum stabil wirkten.
Teilweise wurden auch verschiedene Spannungen bei denselben Frequenzen notiert.
Eine Filterkurve ist nicht zu erkennen, somit sind Aussagen zum Selektivverstärker schwer zu treffen.

Die Stoffe $\ce{Dy2O3}$ und $\ce{Gd2O3}$ wurden verwendet, da die Messungen vor und nach dem Einführen der Probe sichtbare Differenzen haben.
teilweise unterschiedliche Ausgangsspannungen bei denselben 
\section{Anhang}
\label{sec:Anhang}
\begin{figure}
    \centering
    \includegraphics[width=\textwidth]{content/datenselektiv.pdf}
    \caption{Die aufgenommenen Werte für die Messung der Filterkurve des Selektivverstärkers.}
    \label{fig:datenselektiv}
\end{figure}
\begin{figure}
    \centering
    \includegraphics[width=\textwidth]{content/datenmessung.pdf.pdf}
    \caption{Die notierten Werte von der Vermessung der paramagnetischen Proben.}
    \label{fig:datenmessung}
\end{figure}
