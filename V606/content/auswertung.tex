\section{Auswertung}
\label{sec:Auswertung}
\subsection{Untersuchung des Selektivverstärkers}
\label{subsec:Selektivverstäarker}
Der Selektivverstärkers soll zunächst untersucht werden.
Dazu wird die Frequenz und die dazugehörige Ausgangsspannung $U_A$ im Frequenzbereich von 20 bis 40 $\si{\kilo\hertz}$ aufgenommen.
Die Messpaare sind in der Tabelle \ref{tab:Messergebnisse} zu finden.
Zudem ist die Filterkurve in Abbildung \ref{fig:filterkurvePLOT} zu sehen.
Aus der Abbildung \ref{fig:filterkurvePLOT} ist ein Spannungshoch im Frequenzbereich von 22.5 und 25 $\si{\kilo\hertz}$ zu erkennen.
Weitere Aussagen können aus den Messergebnissen nicht getroffen werden.

\begin{table}
  \centering
  \caption{Messergebnisse für die Filterkurve des Selektivverstärkers.}
  \label{tab:Messergebnisse}
  \begin{tabular}{c c}
    \toprule
    {$U_A$ [$\si{\milli\volt}$]} & {$\nu$ [$\si{\micro\ampere}$]}\\
    \midrule
      20.6   & 84 \\
      21.2   & 84 \\
      21.8   & 94 \\
      22.7   & 91 \\
      22.9   & 98 \\
      22.9   & 82 \\
      22.9   & 98 \\
      23.0   & 96 \\
      23.3   & 84 \\
      23.3   & 96 \\
      23.6   & 100 \\
      23.8   & 81 \\
      23.9   & 82 \\
      23.9   & 84 \\
      23.9   & 99 \\
      23.9   & 100 \\
      24.0   & 82 \\
      24.0   & 95 \\
      24.9   & 100 \\
      25.0   & 76 \\
      25.4   & 70 \\
      25.4   & 81 \\
      25.5   & 78 \\
      25.7   & 90 \\
      25.9   & 69 \\
      26.0   & 72 \\
      26.8   & 63 \\
      27.0   & 65 \\
      27.5   & 63 \\
      27.9   & 67 \\
      28.6   & 60 \\
      28.6   & 62 \\
      30.5   & 57 \\
      30.6   & 56 \\
      30.7   & 57 \\
      31.9   & 69 \\
      32.2   & 52 \\
      32.3   & 60 \\
      33.4   & 50 \\
      33.7   & 55 \\
      35.1   & 51 \\
      35.5   & 48 \\
      36.6   & 46 \\
      37.1   & 45 \\
      37.7   & 50 \\
      38.6   & 43 \\
      39.1   & 42 \\
      39.7   & 41 \\
      40.2   & 50 \\
    \bottomrule
  \end{tabular}
\end{table}

\begin{figure}
  \centering
  \caption{Filterkurve des Selektivverstärkers.}
  \label{fig:filterkurvePLOT}
  \includegraphics{content/filterkurvePLOT.pdf}
\end{figure}

\subsection{Bestimmung der Suszeptibilitäten}
\label{subsec:Suszeptibilität}

Die Suszeptibilitäten $\chi$ der beiden Stoffe sollen mithilfe der Messwerte aus Tabelle \ref{tab:Messwerte} bestimmt werden.
Dabei werden die Formeln \eqref{eqn:chi_r} und \eqref{eqn:chi_u} verwendet.
In der Tabelle \ref{tab:Suszeptibilität_Praxis} sind die Ergebnisse zu finden.

\begin{table}
  \centering
  \caption{Messergebnisse zur Bestimmung der Suszeptibilitäten.}
  \label{tab:Messwerte}
  \begin{tabular}{c c c c c c}
    \toprule
    {Stoff} & {$U_{\text{Br ohne}}$ [$\si{\milli\volt}$]} & {$U_{\text{Br mit}}$ [$\si{\milli\volt}$]} & {$R_{3 \text{ohne}}$ [$\si{\milli\ohm}$]} &{$R_{3 \text{mit}}$ [$\si{\milli\ohm}$]} & {$\increment R_3$ [$\si{\milli\ohm}$]}\\
    \midrule
    $\ce{Dy2O3}$ & 0.09 & 0.37 & 2964.5 & 1690.0 & 1274.5\\
              &0.13 & 0.392 & 2981.0 & 1530.0 & 1451\\
              &0.12 & 0.42 & 3059.0 & 1485.5 & 1574\\
    $\ce{Gd2O3}$ &0.20 & 0.405 & 3136.5 & 2360.0 & 776.5\\
              &0.215 & 0.407 & 3157.5 & 2398.0 & 759.5\\
              &0.21 & 0.385 & 3137.0 & 2381.0 & 756\\
    \bottomrule
  \end{tabular}
\end{table}

\begin{table}
  \centering
  \caption{Suszeptibilitäten aus den Messergebnissen.}
  \label{tab:Suszeptibilität_Praxis}
  \begin{tabular}{c c c}
    \toprule
    {Stoff} & {$\chi_U$} & {$\chi_R$} \\
    \midrule
    $\ce{Dy2O3}$ & 0.00867 $\pm$ 0.00029 & 0.02218 $\pm$ 0.00110 \\
    $\ce{Gd2O3}$ & 0.00607 $\pm$ 0.00023 & 0.01217 $\pm$ 0.00008 \\
    \bottomrule
  \end{tabular}
\end{table}

Die Suszeptibilitäten aus den Messergebnissen werden mit dem Theoriewert verglichen.
Dieser wird mithilfe der Formel \eqref{eqn:sus_theo} bestimmt.
In Tabelle \ref{tab:Proben} ist jeweils die Dichte $\rho$, Masse $m$, Länge $l$, molare Masse $M$ und der reale Querschnitt $Q$ der beiden Proben zu finden, die für die Berechnung benötigt werden.
Der reale Querschnitt $Q$ wird mit der Formel \eqref{eqn:qreal} ermittelt.

\begin{table}
  \centering
  \caption{Werte der Proben.}
  \label{tab:Proben}
  \begin{tabular}{c c c c c c }
    \toprule
    {Stoff} & {$\rho$ [$\si{\gram\per\centi\metre\cubed}$]} & {$m$ [$\si{\gram}$]} & {$l$ [$\si{\centi\metre}$]} & {$M$ [$\si{\gram\per\mole}$]} & {$Q$ [$\si{\centi\metre\squared}$]}\\
    \midrule
    $\ce{Dy2O3}$ & 7.8 & 15.1 & 17.3 & 372.9982 & 0.1119\\
    $\ce{Gd2O3}$ & 7.4 & 14.08 & 17.5 & 362.4982 & 0.1087\\
    \bottomrule
  \end{tabular}
\end{table}

\begin{table}
  \centering
  \caption{Quantenzahlen und Landé-Faktoren.}
  \begin{tabular}{c c c c c }
    \toprule
    {Stoff} & {$L$} & {$S$} & {$J$} & {$g_J$}\\
    \midrule
    $\ce{Dy2O3}$ & 5 & 2.5 & 7.5 & 1.33\\
    $\ce{Gd2O3}$ & 0 & 3.5 & 3.5 & 2.0\\
    \bottomrule
  \end{tabular}
\end{table}

\begin{table}
  \centering
  \caption{Vergleich der Suszeptibilitäten.}
  \label{tab:vergleich}
  \begin{tabular}{c c c c c c}
    \toprule
    {Stoff} & {$\chi_T$} &{$\chi_U$} & {$\frac{\chi_T - \chi_U}{\chi_T}$} & {$\chi_R$} & {$\frac{\chi_T - \chi_R}{\chi_T}$}\\
    \midrule
    $\ce{Dy2O3}$ & 0.02569 & 0.00867 $\pm$ 0.00029 & 0.6625 & 0.02218 $\pm$ 0.00110 & 0.1364 \\
    $\ce{Gd2O3}$ & 0.01394 & 0.00607 $\pm$ 0.00023 & 0.5641 & 0.01217 $\pm$ 0.00008 & 0.1269 \\
    \bottomrule
  \end{tabular}
\end{table}