\section{Diskussion}
\label{sec:Diskussion}
Der Plateau-Bereich war nicht eindeutig zu bestimmen.
In der Auswertung beginnt dieser bei $\SI{390}{\volt}$.
Ein Start bei $\SI{440}{\volt}$ wurde auch in Betracht gezogen und hätte zu einer flacheren Plateausteigung geführt.
\\
Die Bestimmung der Totzeit verlief genauer.
Aus beiden Methoden folgten Ergebnisse in derselben Größenordnung.
Mit dem Oszilloskop beträgt die Totzeit $T= \SI{100}{\micro\second}$, aus der Zwei-Quellen-Methode wurde $T= \SI{114 \pm 0.047}{\micro\second}$ bestimmt.
Die Abweichung ist unter $\SI{15}{\percent}$.
\\
Durch die grafische Darstellung wird ersichtlich, dass die Zahl $Z$ mit der mittleren Stromstärke $I$ in einem linearen Zusammenhang steht.
Gleiches ist auch im Zusammenhang zur Ladung $Q$ zu sehen.