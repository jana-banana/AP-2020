\section{Diskussion}
\label{sec:Diskussion}

In dem Versuch wird experimentell die Geschwindigkeit von Schall in Acryl ermittelt. Der Literaturwert ist aus der Quelle \cite{cA} entnommen worden.
Es ergibt sich:
\begin{align*}
    c_{\text{exp}}    &= \SI{2771(20)}{\metre\per\second} \\
    c_{\text{theo}}   &= \SI{2730}{\metre\per\second} \\
    \increment c      &= \SI{1.5}{\percent}
\end{align*}

Außerdem wird der Absorptionskoeffizient $\alpha $ bestimmt zu:
\begin{equation*}
    \alpha = \SI{25.6531(37636)}{\per\metre}
\end{equation*}
Das Ergebnis für die Schallgeschwindigkeit ist recht genau für die Art der Messung. 

\noindent Das Ausmessen des Blockes mit der Schieblehre sollte recht gute Ergebnisse liefern, jedoch war es aufgrund der geringen Durchmesser der Bohrungen schwierig, 
genau den Abstand von der ihren äußersten Stellen zu nehmen. Zu der Aufnahmen der Messpunkte von dem Ultraschallechoskop ist zu sagen, dass auf einem 
Computerbildschirm zwei Cusor nach Augenmaß gelegt werden, welche dann den Punkt bestimmen. 


\noindent Die Untersuchung des Augenmodell stellte die Herausforderung einer gewölbten Oberfläche, sodass bidestilliertes Wasser nicht als Kontaktmittel benutzt 
werden konnte, da es abfloß. Stattdessen wurde Ultraschallgel benutzt. Die gewölbte Fläche brachte auch die Schwierigkeit, dass die Ultraschallsonde sehr ruhig 
gehalten werden musste, da sich die Messergebnisse bei kleinster Bewegung schon veränderten. Somit war das etwas zeitaufwändige Ablesen der Peaks beim Augenmodell
sehr schwierig genau durchzuführen. 